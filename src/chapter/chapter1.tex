\usepackage[utf8]{inputenc}
\usepackage[T1]{fontenc}\chapter*{1. Introducción}
\section*{Antecedentes}
El desarrollo de este Trabajo de Fin de Grado (TFG) surge de la necesidad observada en mi experiencia profesional actual, donde la tarea repetitiva y manual de leer y extraer información de documentos representa una carga significativa.

Este desafío no es exclusivo de mi entorno laboral, sino que es una realidad común en una variedad de sectores incluyendo las compañías de seguros, instituciones educativas, empresas del sector sanitario, empresas de gestión de recursos humanos y entidades financieras entre otras.

Estas organizaciones enfrentan el reto constante de gestionar grandes volúmenes de documentación, lo cual resalta la importancia y la necesidad universal de soluciones automatizadas.

Durante el curso de Proyecto de Ingeniería del Software, tuve la oportunidad de desarrollar una base técnica preliminar que se ha convertido en un pilar fundamental para la realización de este proyecto. Esta implementación basada en la comunicación con Modelos de Lenguaje de Gran Escala (LLMs) en la automatización de procesos de extracción y gestión de información destaca por su capacidad para transformar y optimizar estas tareas.

El uso de LLM en este contexto es relativamente reciente, pero su potencial para entender, generar y manipular lenguaje natural con una precisión revolucionaria los convierte en herramientas excepcionales para superar los desafíos asociados con la gestión de documentación en diversos ámbitos.

Este TFG busca explorar y expandir las capacidades de los LLMs, proporcionando soluciones prácticas y efectivas que pueden ser adoptadas en múltiples sectores para mejorar la eficiencia y la precisión en el tratamiento de la información documental.

\section*{Planteamiento del problema}
En la realidad laboral contemporánea, enfrentamos el desafío de manejar de forma eficaz la información contenida en una vasta cantidad de documentos.

Esta metodología tradicional enfrenta múltiples desafíos significativos:

\begin{itemize}
    \item Alto coste operacional: La gestión manual de documentos consume una cantidad considerable de tiempo, recursos humanos y financieros. El personal dedicado a estas tareas genera un gasto sustancial que impacta directamente en los costos operativos de la organización.
    \item Desviación de recursos valiosos: Los recursos invertidos en la operación manual de documentos podrían ser mejor asignados a actividades que aporten valor real a la organización. Esto incluye tareas que potencian la innovación, el desarrollo estratégico y el servicio al cliente, entre otros.
    \item Incidencia de errores manuales: La tramitación manual de documentos es intrínsecamente susceptible a errores humanos. Estos errores pueden conducir a decisiones basadas en información incorrecta o incompleta.
\end{itemize}

Ante esta situación, se hace imperativo el desarrollo de soluciones tecnológicas que automatizan y optimizan el proceso de extracción de la información.

Este Trabajo de Fin de Grado busca abordar esta problemática desarrollando un sistema automatizado capaz de extraer información de documentos PDF de manera eficiente y precisa utilizando tecnologías avanzadas de procesamiento de lenguaje natural.

\section*{Justificación}
La relevancia de este Trabajo de Fin de Grado (TFG) se fundamenta en la necesidad de implementar soluciones tecnológicas avanzadas que optimicen y automaticen la extracción de la información contenida en documentos.

\begin{itemize}
    \item Desde el punto de vista profesional este proyecto responde a una necesidad práctica de desarrollar una tecnología que permita la recuperación de la información contenida en diferente tipo de documentos.
    \item Desde la perspectiva académica, el desarrollo de un sistema de extracción automática de información aborda competencias clave en la ingeniería informática, tales como la programación, el análisis de sistemas, y especialmente, el procesamiento de lenguaje natural y la inteligencia artificial.
\end{itemize}

En resumen, este TFG no solo es una oportunidad para aplicar habilidades y conocimientos técnicos adquiridos durante el grado, sino también una contribución valiosa a la innovación de sistemas digitales en mi ámbito profesional.

\section*{Objetivos}
El propósito central de este trabajo es la creación de un sistema de automatización para la extracción eficaz de información contenida en documentos PDF.

Los objetivos específicos incluyen:

\begin{enumerate}
    \item Desarrollar una tecnología capaz extraer información de documentos PDF.
    \item Diseñar una interfaz intuitiva que permita a los usuarios interactuar de manera eficiente con esta tecnología.
    \item Evaluar la eficacia del sistema mediante pruebas en diferentes contextos, con el fin de asegurar su adaptabilidad y eficiencia.
\end{enumerate}
