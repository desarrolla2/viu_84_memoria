\begin{resume}
    \textbf{// @TODO: Es una sección concisa y precisa (400 a 500 palabras) que resume los aspectos más importantes del trabajo realizado. Su objetivo es proporcionar una visión general rápida y clara del contenido y los resultados obtenidos, capaz de captar la atención e interés del lector, invitándolo a leer el trabajo completo.
    El resumen debe comenzar con una breve introducción que contextualice el tema del trabajo y exponga la problemática abordada. Debes mencionar de manera breve los objetivos o propósitos principales del trabajo. Brevemente, describe los métodos/metodología o enfoque que has utilizado para llevar a cabo el trabajo. Resume los principales resultados obtenidos. Indica las conclusiones o contribuciones más importantes derivadas de tu trabajo.
    Palabras clave: Al final del resumen, proporciona una lista de palabras clave que representen los conceptos principales y los temas abordados en tu trabajo.
    }

    \vspace{0.5cm}
    \keywords{primero, segundo, tercero}
\end{resume}
