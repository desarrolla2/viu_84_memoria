\chapter{Instrucciones de instalación}\label{ch:appendix_2}

Puedes encontrar una versión actualizada de este documento en la documentación del proyecto en
\textit{github}~\cite{url_viu_84_proyecto_install}.

\section*{Instalar dependencias del proyecto}\label{sec:instalar_dependencias}

Para la ejecución de este proyecto, es necesario tener instalados los paquetes que aparecen indicados en la
tabla~\ref{tab:dependencies} junto con sus versiones mínimas.

\begin{table}[h]
    \renewcommand{\arraystretch}{1.5}
    \setlength{\tabcolsep}{10pt}
    \begin{tabular}{>{\bfseries}p{0.75\textwidth} >{\raggedleft\arraybackslash}p{0.15\textwidth}}
        \toprule
        \textbf{Paquete}                         & \textbf{Versión mínima} \\
        \midrule
        docker~\cite{url_docker}                 & 20.10                   \\
        docker-compose~\cite{url_docker_compose} & 1.25                    \\
        git~\cite{url_git}                       & 2.34                    \\
        \bottomrule
    \end{tabular}
    \caption{Dependencias del proyecto y versiones mínimas}
    \label{tab:dependencies}
\end{table}

\section*{Proceso de instalación}\label{sec:proceso_instalacion}

\subsection*{Descargar el código fuente}\label{subsec:descargar_codigo}

Descargamos el código fuente desde el repositorio principal del proyecto.

\begin{lstlisting}[language=bash]
git clone git@github.com:desarrolla2/viu_84_trabajo_fin_grado
\end{lstlisting}

\subsection*{Configurar el entorno docker}\label{subsec:configurar_docker}

En este paso será necesario copiar archivos de configuración de docker, para ello ejecutamos lo siguiente:

\begin{lstlisting}[language=bash]
cp docker/.env.dist .env
cp docker/docker-compose.yml.dist docker-compose.yml
cp docker/Makefile.dist Makefile
\end{lstlisting}

Si también se desea contar con los contenedores ELK descritos en la subsección \ref{subsec:logs} Registros, para la
visualización de los logs deberás ejecutar:

\begin{lstlisting}[language=bash]
cp docker/docker-compose.override.yml.dist docker-compose.override.yml
\end{lstlisting}

Finalmente, puedes reescribir la configuración si fuera necesario para adecuarla a tus necesidades.

\subsection*{Construir contenedores e instalar dependencias}\label{subsec:construir_contenedores}

\subsubsection*{Construir contenedores}\label{subsubsec:construir_contenedores}

Para la construcción de los contenedores hay una tarea definida en el fichero \textit{MakeFile}

\begin{lstlisting}[language=bash]
make docker-build
\end{lstlisting}

\subsubsection*{Instalar dependencias PHP}\label{subsubsec:instalar_dependencias_php}

También hay una tarea definida para la instalación de las dependencias.

\begin{lstlisting}[language=bash]
make php-composer-install
\end{lstlisting}

\subsubsection*{Levantar los contenedores}\label{subsubsec:levantar_contenedores}

La siguiente tarea levanta los contenedores necesarios para la ejecución del proyecto.

\begin{lstlisting}[language=bash]
make docker-up
\end{lstlisting}

\subsubsection*{Editar archivo .env}\label{subsubsec:editar_env}

Finalmente, edita el archivo \textit{.env} para actualizar los siguientes valores:

\begin{itemize}
    \item \textbf{APP\_SECRET}: establece un token único.
    \item \textbf{OPENAI\_AUTHENTICATION\_TOKEN}: genera un token válido a través de la plataforma de \textit{OpenAI}~
    \cite{url_openai_platform}
\end{itemize}


\section*{Ejecutar pruebas}\label{sec:ejecutar_pruebas}

Una vez el proyecto está correctamente instalado, puedes ejecutar la batería de pruebas a través de la siguiente tarea.

\begin{lstlisting}[language=bash]
make tests
\end{lstlisting}