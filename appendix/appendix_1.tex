\chapter{Instrucciones de instalación}\label{ch:appendix_1}


El apéndice es toda la información adicional complementaria, necesaria para ilustrar mejor el cuerpo del trabajo.

Los apéndices van numerados con una secuencia alfabética de letras mayúsculas (A, B, C, etc.). En caso de que se hayan
producido diagramas de clases, diagramas de la base de datos, casos de usos u otra información gráfica y éstos no hayan
sido incluidos como parte del cuerpo del trabajo, deberán ser incluidos en la sección de apéndices.


Apéndice A

Ejemplos formatos
El estilo del párrafo tiene que ser encuadrado (justificado).
En caso de terminar con un apartado (nivel uno de título) aplicar un salto de página para comenzar siempre el siguiente
apartado en una página nueva.

Tablas

Tanto las figuras como las tablas tienen que estar indicadas en el texto y referenciadas con la numeración que tiene
asignada. La descripción debe ser clara y explicar lo que se quiere representar con independencia de la referencia del
texto.

Tanto las tablas como las figuras tendrán un estilo de párrafo centrado. La descripción se realizará desde “Insertar
título”.


Tabla 1. Operaciones matemáticas utilizadas en el estudio realizado. Elaboración propia.
operación
símbolo
ejemplo
suma
+
4 + 4 = 8
resta
-
4 - 2 = 2
multiplicación
*
4 * 4 = 16

Figuras

Las figuras deben estar mencionadas en el texto y ser referenciadas por su numeración Ejemplo: Se habilita un servidor
con Jupyter como se puede ver en la Figura 1, en el cual se puede disponer de diferentes kernels de programación
(Python, R, Julia…).

Figura 1. Arquitectura Jupyter Cliente - Servidor.
Fuente: https://www.paradigmadigital.com/dev/jupyter-data-science-aplicada/


Viñetado

Está permitido el uso de viñetado:

• Esta sería la forma
• Segundo
◦ En caso de tener un sub apartado
Una numeración que se quiera hacer no debe influir en la numeración de los apartados generales:

1. Item 1
2. Item 2
