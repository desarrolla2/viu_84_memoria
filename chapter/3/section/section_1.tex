\section{Metodologías ágiles}

En este proyecto se utilizaron las siguientes metodologías enmarcadas dentro de las metodologías ágiles.
La \textit{programación extrema} introdujo los ciclos de desarrollo cortos y frecuentes, así como la comunicación
fluida con los miembros del equipo, en este caso mi directora de TFG.
El \textit{desarrollo dirigido por pruebas} ayudó a obtener un diseño óptimo y bien testeado, mientras que la
\textit{metodología iterativa incremental} permitió evaluar los avances en pequeños pasos, permitiendo tomar
decisiones rápidamente cuando fue necesario.

\subsection{Programación extrema}

La Programación Extrema o \textit{extreme programming} (XP) es una metodología ágil de desarrollo de software que se
enfoca en mejorar la calidad del software y la capacidad de respuesta a las necesidades cambiantes del cliente.
Introducida por Kent Beck en el libro \textit{Extreme Programming Explained: Embrace Change}~\cite{book_beck_1999},
XP promueve la colaboración intensa entre los desarrolladores y los clientes, ciclos de desarrollo cortos y frecuentes,
y la entrega continua de pequeñas mejoras.

\subsection{Desarrollo dirigido por pruebas}

El desarrollo dirigido por pruebas o \textit{test driven development} (TDD) es una práctica de desarrollo de software.
Este enfoque fue popularizado por Kent Beck, uno de los pioneros de las metodologías ágiles, en su libro
\textit{Test Driven Development: By Example}~\cite{book_beck_2003}.
TDD se basa en ciclos cortos de desarrollo donde se escribe una prueba, se implementa el código necesario para pasar la
prueba y luego se reescribe el código para mejorar su estructura sin cambiar su comportamiento.

\subsection{Iterativo incremental}\label{subsec:iterativo_incremental}

El enfoque iterativo incremental es una metodología utilizada en el desarrollo de software que combina dos modelos:
el iterativo y el incremental.
Esta metodología es popular en el desarrollo ágil y se centra en desarrollar un sistema a través de ciclos repetidos
(iteraciones) y en la construcción gradual de funcionalidad (incrementos).

El término ``iterativo'' se refiere a la repetición de un conjunto de actividades a lo largo del ciclo de vida del
desarrollo del software.
En cada iteración, se planea, desarrolla y evalúa una parte del sistema.

El término ``incremental'' se refiere a la construcción del sistema mediante adiciones sucesivas de componentes y
funcionalidades.
Cada incremento agrega una parte funcional del sistema hasta que el producto está completo.