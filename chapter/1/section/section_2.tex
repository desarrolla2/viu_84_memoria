\section{Planteamiento del problema}

Supongamos una empresa que gestiona seguros de coche, deberá recibir un paquete de datos de cada nuevo cliente que
contendrá entre otros los siguientes documentos:

\begin{itemize}
    \item Documentos de identidad del titular y los tomadores
    \item Permiso de conducción de los tomadores
    \item Ficha técnica y permiso de circulación del vehículo
    \item Recibo del impuesto de vehículo de tracción mecánica
\end{itemize}

La forma tradicional de obtener la información que contienen dichos documentos consiste en que un operario reciba los
documentos, los abra y los introduzca en el sistema.
Esta metodología tradicional enfrenta una problemática significativa:

\begin{itemize}
    \item \textbf{Elevado coste}:
    el personal dedicado a estas tareas genera un gasto que impacta directamente en el coste operativo de la
    organización.
    \item \textbf{Demora en los tiempos de tramitación}:
    la tramitación manual implica que los documentos no van a ser procesados en el momento en que son recibidos, sino
    que deberán esperar a que un operario esté disponible para ocuparse de esta tarea.
    \item \textbf{Pobre asignación de recursos}:
    los recursos invertidos en la tramitación manual de documentos podrían ser asignados a actividades que aporten un
    mayor valor a la organización.
    Esto incluye tareas como la innovación, el desarrollo estratégico y el servicio al cliente, entre otros.
    \item \textbf{Errores manuales}:
    en la tramitación manual de documentos es intrínsecamente susceptible a los errores humanos.
\end{itemize}

Ante esta situación, se hace necesario el desarrollo de soluciones responsables de automatizar el proceso de extracción
de la información.