\section{Planteamiento del problema}\label{sec:plantenamiento_problema}

Supongamos una empresa que gestiona seguros de coche, para dar de alta un nuevo cliente deberá recibir y procesar
un paquete de datos que contendrá entre otros los siguientes documentos:

\begin{itemize}
    \item Documentos de identidad del titular y los tomadores
    \item Permiso de conducción de los tomadores
    \item Ficha técnica y permiso de circulación del vehículo
    \item Recibo del impuesto de vehículo de tracción mecánica
\end{itemize}

La forma tradicional de tramitar la información que contienen estos documentos consiste en que un operario reciba los
documentos, los abra e introduzca los datos en el sistema.
Esta metodología enfrenta la siguiente problemática:

\begin{itemize}
    \item
    El personal dedicado a estas tareas genera un \textbf{elevado coste} que impacta directamente en los costes
    operativos de la organización.

    \item
    La tramitación manual implica la \textbf{demora en los tiempos de tramitación} de documentos que no van a ser
    procesados en el momento en que son recibidos, sino que deberán esperar a que un operario esté disponible para
    ocuparse de esta tarea.

    \item
    \textbf{Pobre asignación de recursos humanos} que podrían ser asignados a actividades que aporten un mayor valor a
    la organización.
    Esto incluye tareas como la innovación, el desarrollo estratégico y el servicio de atención al cliente, entre otros.

    \item
    \textbf{Errores manuales} en la tramitación de documentos.
\end{itemize}

Ante esta situación, se hace necesario el desarrollo de soluciones con capacidades de automatizar el proceso de
extracción de la información de documentos.
