\section{Trabajo futuro}

A medida que la tecnología y las necesidades del mercado evolucionan, siempre existen oportunidades para mejorar y
extender el trabajo realizado.

En este contexto, se presentan algunas sugerencias y recomendaciones para futuros trabajos en el ámbito de este proyecto
de extracción de información de documentos.

\subsection*{Aumentar el conjunto de datos de prueba}

Si bien el conjunto de datos de prueba es adecuado para el alcance de este TFG, es necesario ampliarlo para obtener
conclusiones más relevantes sobre la precisión de la respuesta.
Utilizar un conjunto de datos más grande y diverso permitirá evaluar mejor la robustez del sistema y su capacidad para
manejar diferentes tipos de documentos y situaciones.

Esto ayudará a asegurar que el sistema sea confiable y efectivo en un rango más amplio de aplicaciones.

Otra mejora en este aspecto es designar un equipo externo que se ocupe de la evaluación de las pruebas para evitar que
se produzca el sesgo del creador, un sesgo que se produce cuando la misma persona que desarrolla el código también lo
prueba, lo que puede resultar en la falta de detección de errores~\cite{url_test_io_bias_software_testing}.

\subsection*{Aumentar el soporte a otros formatos}

Aunque una gran parte de la documentación gestionada está en formato \textit{PDF}, es conveniente incrementar el soporte
para otros formatos igualmente comunes.

Por ejemplo se podría utilizar \textit{LibreOffice}~\cite{url_libreoffice} para convertir documentos en formato
\textit{word}~\cite{url_microsoft_word} o en formato \textit{excel}~\cite{url_microsoft_excel} a texto.

Además, se podría emplear \textit{Tesseract OCR}~\cite{url_tesseract}, una herramienta que permite convertir
imágenes en texto, utilizando una tecnología de reconocimiento óptico de caracteres u
\textit{optical character recognition} (OCR) para convertir documentos escaneados o fotografías a texto.

\subsection*{Probar con otros modelos}

La rápida evolución de los modelos de lenguaje natural ofrece continuas oportunidades para mejorar la precisión, los
tiempos de respuesta y el costo por documento procesado.
Aunque en este trabajo se ha utilizado \textit{ChatGPT 4}, es importante seguir probando nuevos modelos a medida que se
desarrollan.

Evaluar estos modelos en términos de precisión, eficiencia y costo ayudará a identificar cuál es el más adecuado para
diferentes casos de uso, asegurando que el sistema se mantenga a la vanguardia de la tecnología y sea capaz de ofrecer
el mejor rendimiento posible.

\subsection*{Utilizar otras técnicas para extraer la información}

Aunque los modelos LLM ofrecen una solución óptima para documentos con formato desconocido, para otros documentos el
formato será conocido y tendrá pocas variaciones, como por ejemplo una declaración de la renta, facturas comerciales o
los recibos de servicios públicos.

En estos casos, es más eficiente y económico extraer la información mediante técnicas más rudimentarias, como por
ejemplo las expresiones regulares.
Estas técnicas permiten identificar y extraer datos específicos de manera rápida y con un menor costo, aprovechando la
estructura predecible de los documentos.

\subsection*{Incrementar la seguridad de los datos}

La privacidad y la seguridad de los datos son aspectos críticos en cualquier sistema de procesamiento de información,
especialmente cuando se manejan documentos sensibles.

La incorporación de prácticas de seguridad desde el diseño, conocido como \textit{privacy by design}, puede ayudar a
construir un sistema más seguro en este sentido.
Para ello, se podrían implementar políticas de acceso, cifrado de datos y anonimización de los mismos cuando sea
adecuado.

Estas medidas no solo protegen la información personal durante el procesamiento y almacenamiento, sino que también
aseguran el cumplimiento de normativas de protección de datos.


\subsection*{Liberar el proyecto}

Publicar el proyecto como código abierto permitirá que la comunidad de desarrolladores contribuya al mejoramiento y
extensión del sistema.
Esto puede incluir la identificación y corrección de errores y el soporte para nuevos tipos de formatos o documentos,
entre otros.

Para ello se propone crear una documentación detallada sobre cómo contribuir al proyecto, incluyendo guías de estilo
de código, procesos de revisión y pautas para la presentación de issues y pull requests.

Además, será necesario seleccionar una licencia adecuada para el proyecto que defina claramente los derechos y
responsabilidades de los usuarios y contribuyentes.
Licencias como MIT, Apache 2.0 o GNU GPL son opciones comunes que pueden ser consideradas.

\subsection*{Desplegar en producción}

Desplegar el sistema en un entorno de producción es un paso vital que pone a prueba la capacidad del proyecto para
operar bajo condiciones reales y cumplir con las expectativas del usuario final.
Esto permite identificar y resolver problemas que pueden no haber sido evidentes en un entorno de desarrollo o
pruebas controladas.

Este proceso no solo válida la funcionalidad y la eficiencia del sistema, sino que también proporciona oportunidades
para optimizar, asegurar y mejorar continuamente el sistema.