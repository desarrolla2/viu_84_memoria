\section{Recapitulación de los objetivos}

Al inicio de este proyecto en la sección~\ref{sec:objetives} Objetivos, se establecieron tres objetivos principales,
cada uno orientado a abordar diferentes aspectos del proceso de extracción de información automatizada desde documentos.

A continuación, se presenta una recapitulación de cada uno de estos objetivos y su grado de cumplimiento:

El \textbf{Requisito 1} era desarrollar un sistema capaz de convertir documentos \textit{PDF} en texto plano.
Este objetivo se logró con éxito mediante el desarrollo del componente \textit{Generator}.
Este componente utiliza la herramienta \textit{Pdf to Text} para realizar la conversión.

El \textbf{Requisito 2} fue extracción de información de contratos de arrendamiento de vivienda entre particulares.
Este objetivo se cumplió mediante el desarrollo del componente \textit{Reader} que utiliza la \textit{API}
de \textit{ChatGPT} interpretar y extraer información de los documentos convertidos.
Las pruebas realizadas determinaron que el sistema puede extraer con precisión el 100\% de la información relevante
en los contratos de arrendamiento.

Finalmente, el \textbf{Requisito 3} consistía extracción de información de contratos de compraventa de vehículo entre
particulares.
Al igual que con los contratos de arrendamiento, el componente \textit{Reader} se encargó de este objetivo.

Finalmente, destacar que el diseño modular implementado permite incorporar fácilmente soporte para nuevos formatos de y
tipos de documentos.