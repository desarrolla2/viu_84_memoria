\section{Síntesis de los resultados}

Durante el desarrollo del proyecto se alcanzaron resultados relevantes que demuestran la relevancia del sistema
desarrollado para la extracción de información de documentos.
A continuación, se muestra un resumen los principales resultados:

\textbf{Desarrollo de un sistema de extracción de información}

Se implementó un sistema dividido en dos componentes principales: \textit{Generator} y \textit{Reader}.
El componente \textit{Generator} convierte documentos PDF en texto plano, mientras que el componente \textit{Generator}
extrae información relevante del texto procesado.

Dentro de cada componente se diseñó un sistema de preprocesadores, procesadores y postprocesadores que permiten
añadir fácilmente soporte para nuevos tipos de formato o de documentos.

\textbf{Rendimiento y Eficiencia}

En este proyecto, el rendimiento y la eficiencia no son apartados trascendentales, siempre y cuando el procesamiento de
documentos se mantenga en el rango de unos pocos segundos.
El núcleo del sistema es rápido y eficiente, pero el rendimiento general depende en gran medida de las herramientas
seleccionadas para la infraestructura.

\begin{enumerate}
    \item \textbf{PDF To Text}: Esta herramienta, que se ejecuta localmente, ofrece un buen rendimiento debido a su
    rapidez en el procesamiento de documentos PDF.
    \item \textbf{API de ChatGPT}
    El rendimiento de esta herramienta dependerá significativamente del tipo de documento y de la
    carga del sistema en un momento concreto.
    En general, el rendimiento no será tan bueno como el de las herramientas locales.
\end{enumerate}


\textbf {Calidad del código y mantenibilidad}

\textbf{Calidad del Código y mantenibilidad}

La calidad del código y su mantenibilidad en este proyecto han sido elevadas gracias a la adopción de estándares de la
industria, como los sugeridos por Robert C. Martin en ``Código Limpio'' y ``Arquitectura Limpia''.
Estos estándares aseguran que el código sea robusto, consistente y fácil de mantener a lo largo del tiempo.
La buena cobertura de pruebas en la capa de dominio, con un 72.41\%, refleja un enfoque riguroso en la validación de la
lógica de negocio, mientras que la capa de infraestructura, con un 42.86\%, aunque moderada, indica es un área a la
que también se le ha prestado atención.

El análisis estático del código, realizado mediante Scrutinizer-CI, ha proporcionado una puntuación perfecta de diez
sobre diez, destacando la adherencia a las mejores prácticas y la ausencia de errores críticos.

La documentación detallada del proyecto, incluida en este TFG asegura su mantenibilidad a largo plazo.

\textbf {Precisión en la extracción de datos}

El sistema demostró una precisión del 100\% en la identificación y extracción de datos para todos los conjuntos de
prueba.
Si bien los conjuntos de prueba eran relativamente pequeños, este nivel de precisión resalta la robustez del sistema y
su capacidad para manejar diversos tipos de documentos con alta exactitud.

\textbf {Pruebas e integración continua}

Se implementaron prácticas de integración continúa utilizando GitHub Actions, lo que permitió automatizar la ejecución
de pruebas.
Esta integración asegura que cualquier cambio en el código sea rigurosamente probado y validado.

\textbf {Adaptabilidad y extensibilidad}

El sistema fue diseñado con una arquitectura limpia, lo que facilita la incorporación de nuevos tipos de documentos y
métodos de procesamiento.
Esta flexibilidad asegura que el sistema pueda adaptarse a diferentes necesidades, manteniendo su relevancia y utilidad.