\chapter{Conclusiones}\label{ch:chapter_6}




\section{Iteraciones y mejoras}
\colorbox{color_highlight}{@TODO:}

Si has realizado iteraciones o mejoras en el proceso de desarrollo, menciona los cambios realizados y las razones detrás
de ellos.

Nuestro próximo objetivo es abrir el proyecto a la comunidad, convirtiéndolo en una iniciativa de código abierto.

Esto no solo incluirá la mejora y ampliación de la documentación existente, sino también la traducción al inglés para
facilitar su adopción y contribución global.
Además, planeamos desarrollar más funcionalidades y realizar pruebas exhaustivas para asegurar la robustez y fiabilidad
del software.


\section{Teseract}


\section{Rubrica}
\colorbox{color_highlight}{@TODO:}
Revisar al rubrica.

Perdona Silvana que creo que no te contestamos ninguno. Antes de nada, ¡gracias!



Veo que tu correo va focalizado al proceso de registro de la marca, pero no sé si podrías informarte de
\textbf{\textit{qué supone el registro de la marca.}}
Es decir, ¿qué ganamos con ello? ¿merece la pena o no?, pues tenía entendido que por un lado está la marca registrada
pero luego puedes tener diferentes nombres comerciales (no sé si esto es cierto). Además, en caso de querer registrar la
marca, ¿deberíamos registrar los nombres de todas las sociedades? ¿Es suficiente con un nombre genérico? ¿has podido
confirmar lo que comentabas de que en el buscador te aparecía ya la marca Stronghold?


Vamos comentando


