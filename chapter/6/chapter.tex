    \chapter{Conclusiones}\label{ch:chapter_6}

\colorbox{color_highlight}{@TODO: Este capítulo no está desarrollado todavía}

Las conclusiones deben ser claras, concisas y basadas en los resultados y la evidencia presentada en tu trabajo.
Evita introducir nueva información en esta sección y enfócate en
resumir los aspectos más destacados y significativos de tu trabajo.


\section{Recapitulación de los objetivos}

Repasa brevemente los objetivos establecidos al comienzo del y evalúa en qué medida se
han logrado.

Destaca los principales resultados y contribuciones de tu proyecto.


\section{Síntesis de los resultados}

Resume los resultados más importantes y relevantes obtenidos durante el desarrollo.


\section{Implicaciones y relevancia}


Discute las implicaciones de tus resultados y su relevancia para el campo
de la ingeniería informática.
Explica cómo tus hallazgos contribuyen al conocimiento existente o tienen aplicaciones prácticas.
Destaca el valor y la importancia de tu trabajo.


\section{Reflexión sobre el proceso de desarrollo}

Realiza una reflexión sobre el proceso de desarrollo de tu proyecto y evalúa
su eficacia.
Identifica las fortalezas y las limitaciones de tu enfoque y ofrece recomendaciones para futuros proyectos similares.
Describe cualquier desafío o dificultad que hayas enfrentado durante el desarrollo y cómo los has abordado.


\section{Lecciones aprendidas}

Comparte las lecciones aprendidas durante tu trabajo y desarrollo.
Destaca las habilidades y conocimientos adquiridos, así como los aspectos que te han permitido crecer como ingeniero
informático.
Reflexiona sobre los desafíos, los éxitos y los obstáculos superados durante el proyecto.


\section{Trabajo futuro}

Proporciona sugerencias y recomendaciones para futuros trabajos o investigaciones en el campo.
Identifica áreas que requieren una mayor exploración, mejoras adicionales o extensiones de tu trabajo actual.


\section{Rubrica}
\colorbox{color_highlight}{@TODO:Revisar al rubrica.}

