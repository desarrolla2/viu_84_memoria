\chapter{Conclusiones}\label{ch:chapter_6}


\section{Recapitulación de los objetivos}

Al inicio de este proyecto, se establecieron tres objetivos principales, cada uno orientado a abordar diferentes
aspectos del proceso de extracción de información automatizada desde documentos.

A continuación, se presenta una recapitulación de cada uno de estos objetivos y su grado de cumplimiento:

\textbf{Requisito 1: desarrollar un sistema capaz de convertir documentos PDF en texto plano}

Este objetivo se logró con éxito mediante el desarrollo del componente \textit{Generator}.
Este componente utiliza la herramienta \textbf{Pdf to Text} para realizar la conversión.

\textbf{Requisito 2: extracción de información de contratos de arrendamiento de vivienda entre particulares}

Este objetivo se cumplió mediante el desarrollo del componente \textit{Reader} que utiliza la \textbf{API}
de \textbf{ChatGPT} interpretar y extraer información de los documentos convertidos.
Las pruebas realizadas demostraron que el sistema puede extraer con precisión el 100\% de la información relevante
en los contratos de arrendamiento.

\textbf{Requisito 3: extracción de información de contratos de compraventa de vehículo entre particulares}

Al igual que con los contratos de arrendamiento, el componente \textit{Reader} se encargó de este objetivo.

Finalmente, que el diseño modular implementado permite incorporar fácilmente soporte para nuevos formatos de
documentos y nuevos tipos de documentos.


\section{Síntesis de los resultados}

Durante el desarrollo del proyecto se alcanzaron resultados relevantes que demuestran la efectividad y relevancia del
sistema desarrollado para la extracción de información de documentos.
A continuación, se muestra un resumen los principales hallazgos:

\textbf{Desarrollo de un Sistema de Extracción de Información}

Se implementó un sistema dividido en dos componentes principales: \textit{Generator} y \textit{Reader}.
El componente \textit{Generator} convierte documentos PDF en texto plano, mientras que el componente \textit{Generator}
extrae información relevante del texto procesado.

Dentro de cada componente se diseñó un sistema de preprocesadores, procesadores y postprocesadores que permiten
añadir fácilmente nueva funcionalidad.

Este diseño modular permite la fácil extensión y adaptación del sistema para diferentes tipos de documentos.

\textbf {Rendimiento y Eficiencia}

\colorbox{color_highlight}{@TODO:Desarrollar.}

- el nucleo es rápido.
- depende de erramientas de infraestrucutra.
- las que se ejecutan en local tienen buen rendimiento
- las que se ejecutan en remoto, un mal rendimiento.

\textbf {Calidad del Código y Mantenibilidad}

\colorbox{color_highlight}{@TODO:Desarrollar.}


\textbf {Precisión en la extracción de datos}

El sistema demostró una precisión del 100\% en la identificación y extracción de datos para todos los conjuntos de
prueba, desarrollado.
Este alto nivel de precisión resalta la robustez del sistema y su capacidad para manejar diversos tipos de documentos
con alta exactitud.

\textbf {Pruebas e integración continua}

Se implementaron prácticas de integración continúa utilizando GitHub Actions, lo que permitió automatizar la ejecución
de pruebas.
Esta integración asegura que cualquier cambio en el código sea rigurosamente probado y validado.

\textbf {Adaptabilidad y extensibilidad}

El sistema fue diseñado con una arquitectura limpia, lo que facilita la incorporación de nuevos tipos de documentos y
métodos de procesamiento.
Esta flexibilidad asegura que el sistema pueda adaptarse a diferentes necesidades, manteniendo su relevancia y utilidad.


\section{Implicaciones y relevancia}

El desarrollo y la implementación del sistema de extracción de información de documentos tiene varias implicaciones
significativas y una relevancia considerable en múltiples áreas, tanto en el ámbito profesional como en el académico.


\colorbox{color_highlight}{@TODO:Desarrollar.}


\section{Reflexión sobre el proceso de desarrollo}

El proceso de desarrollo de este proyecto ha sido una experiencia enriquecedora y formativa, proporcionando valiosas
lecciones en diversas áreas de la ingeniería del software y la gestión de proyectos.
A continuación, se presenta una reflexión sobre los aspectos más destacados y los desafíos enfrentados durante el
desarrollo de este TFG.

\begin{enumerate}
    \item \textbf{Metodologías ágiles}
    La planificación y ejecución de sprints requirieron una constante evaluación y ajuste, lo que implicó un esfuerzo
    adicional.
    Sin embargo, este enfoque resultó en una mayor flexibilidad y capacidad de respuesta a las necesidades emergentes
    del proyecto.
    \item \textbf{Arquitectura limpia}
    La implementación de esta metodología aseguró que el núcleo del sistema no estuviera acoplado a ninguna
    herramienta externa, permitiendo intercambiar unas por otras y generando un sistema altamente modular y extensible.
    \item \textbf{Desarrollo dirigido por pruebas}
    La implementación de esta metodología aseguró que cada componente del sistema fuera rigurosamente evaluado desde el
    principio.
    Esto no solo mejoró la calidad del código, sino que también facilitó la identificación y corrección temprana de
    errores.
    \item \textbf{Integración de tecnologías de última generación}
    La integración de tecnologías avanzadas como modelos de lenguaje de gran escala (LLM), en este caso a través de
    la API de \textbf{ChatGPT} para la evaluación de documentos, \textbf{Docker} para la contenerización, \textbf{ELK}
    para el registro de logs entre otras, permitió desarrollar un sistema robusto y novedoso.
    Sin embargo, La curva de aprendizaje asociada con la adopción de nuevas tecnologías y herramientas fue
    significativa.
    La configuración de estas herramientas para que funcionaran correctamente requirió un tiempo y esfuerzo
    considerable.
\end{enumerate}

En conclusión, el proceso de desarrollo de este proyecto ha sido un recorrido desafiante pero gratificante, lleno de
aprendizajes y crecimiento.


\section{Trabajo futuro}

A medida que la tecnología y las necesidades del mercado evolucionan, siempre existen oportunidades para mejorar y
extender el trabajo realizado.

En este contexto, se presentan algunas sugerencias y recomendaciones para futuros trabajos en el ámbito de este proyecto
de extracción de información de documentos.

\subsection*{Liberar el proyecto}

Publicar el proyecto como código abierto permitirá que la comunidad de desarrolladores contribuya al mejoramiento y
extensión del sistema.
Esto puede incluir la identificación y corrección de errores y el soporte para nuevos tipos de formatos o documentos,
entre otros.

Para ello se propone crear una documentación detallada sobre cómo contribuir al proyecto, incluyendo guías de estilo
de código, procesos de revisión y pautas para la presentación de issues y pull requests.

Además, será necesario seleccionar una licencia adecuada para el proyecto que defina claramente los derechos y
responsabilidades de los usuarios y contribuyentes.
Licencias como MIT, Apache 2.0 o GNU GPL son opciones comunes que pueden ser consideradas.

\subsection*{Desplegar en producción}

Desplegar el sistema en un entorno de producción es un paso vital que pone a prueba la capacidad del proyecto para
operar bajo condiciones reales y cumplir con las expectativas del usuario final.
Esto permite identificar y resolver problemas que pueden no haber sido evidentes en un entorno de desarrollo o
pruebas controladas.

Este proceso no solo válida la funcionalidad y la eficiencia del sistema, sino que también proporciona oportunidades
para optimizar, asegurar y mejorar continuamente el sistema.















