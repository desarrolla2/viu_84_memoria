\section{Análisis de los resultados}

Para terminar evaluaremos la precisión de los datos obtenidos.
Se realizó una revisión manual de todos los resultados obtenidos, confirmando que en todos los casos fueron
correctos.

En la tabla~\ref{tab:data_set_performance} puede verse que el sistema fue capaz de identificar y extraer
correctamente el 100\% de los datos para todos los conjuntos.

\begin{table}[h]
    \renewcommand{\arraystretch}{1.5}
    \setlength{\tabcolsep}{10pt}
    \begin{tabular}{>{\bfseries}p{0.35\textwidth} p{0.15\textwidth} p{0.15\textwidth} p{0.15\textwidth}}
        \toprule
        \textbf{Conjunto} & \textbf{Documentos} & \textbf{Correctos} & \textbf{Precisión} \\
        \midrule
        \textbf{000}      & 3                   & 3                  & 100 \%             \\
        \textbf{001}      & 9                   & 9                  & 100 \%             \\
        \textbf{002}      & 9                   & 9                  & 100 \%             \\
        \bottomrule
    \end{tabular}
    \caption{Evaluación de los resultados por tipo de conjunto}
    \label{tab:data_set_performance}
\end{table}

Sin embargo, dado que los modelos LLM y en concreto el modelo \textbf{ChatGPT} usado es un sistema no determinista y
que la muestra utilizada en esta evaluación es relativamente pequeña, es posible que la precisión real del sistema sea
ligeramente inferior en escenarios más amplios y variados.
La naturaleza no determinista de \textbf{ChatGPT}
implica que los resultados pueden variar entre ejecuciones, lo que debe ser considerado al interpretar estos resultados.
