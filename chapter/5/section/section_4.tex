\section{Análisis de los resultados}

\colorbox{color_highlight}{@TODO: Esta sección no está desarrollada todavía}

* Pruebas y resultados funcionales: se presentan los resultados de las pruebas realizadas para verificar el
funcionamiento correcto de la solución. Muestra cómo se han llevado a cabo las pruebas y los casos de prueba
utilizados. Destaca los resultados obtenidos en términos de la funcionalidad y el cumplimiento de los requisitos
establecidos.

* Casos de estudio o resultados específicos: Si has realizado estudios de casos específicos o evaluaciones particulares,
describe los resultados obtenidos y su relevancia para tu proyecto. Puedes incluir ejemplos concretos de cómo la
solución informática ha sido aplicada en situaciones reales y los resultados obtenidos en cada caso.

* Resultados cuantitativos: Presenta los resultados numéricos o medibles de tu proyecto. Esto puede incluir métricas de
rendimiento, tiempos de respuesta, velocidad de procesamiento, eficiencia, precisión, entre otros. Utiliza tablas,
gráficos u otros medios visuales para mostrar claramente los datos recopilados.

* Resultados cualitativos: Si tu proyecto implica evaluaciones subjetivas o cualitativas, como la usabilidad, la
experiencia del usuario o la calidad percibida, describe los resultados obtenidos a través de encuestas, entrevistas o
pruebas de usabilidad.

* Validación y pruebas: Explica cómo se han validado y evaluado los resultados de tu proyecto. Si has realizado pruebas,
verifica que se cumplan los requisitos establecidos y describe cómo se ha llevado a cabo la evaluación. Si se han
utilizado conjuntos de datos de prueba o casos de uso específicos, menciónalos en esta sección.
* Comparación con resultados esperados: Compara tus resultados con los objetivos y las expectativas establecidos en la
introducción de tu trabajo. Destaca si has logrado alcanzar tus metas y si los resultados obtenidos son consistentes
con las hipótesis o predicciones iniciales. Si hay desviaciones o discrepancias, explícalas y proporciona posibles
explicaciones.

* Análisis de los resultados: Realiza un análisis de los resultados obtenidos y su relevancia para tu proyecto.

* Limitaciones y posibles mejoras: Menciona las limitaciones o restricciones que puedan haber afectado tus resultados.
Esto puede incluir limitaciones en los datos, en los métodos utilizados o en la implementación del proyecto. También
puedes sugerir posibles mejoras o áreas de investigación futuras basadas en las limitaciones identificadas.