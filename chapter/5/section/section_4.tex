\section{Análisis de los resultados}

Para terminar evaluaremos la precisión de los datos obtenidos, tratándose de este caso de un 100\%
de precisión en cuanto a los resultados previstos.

En la tabla~\ref{tab:data_set_performance} puede verse que el sistema fue capaz de identificar y extraer
correctamente el 100\% de los datos para todos los conjuntos.

\begin{table}[h]
    \renewcommand{\arraystretch}{1.5}
    \setlength{\tabcolsep}{10pt}
    \begin{tabular}{>{\bfseries}p{0.10\textwidth} p{0.55\textwidth} p{0.15\textwidth}}
        \toprule
        \textbf{Conjunto} & \textbf{Contenido}                                        & \textbf{Precisión} \\
        \midrule
        \textbf{000}      & Ruido                                                     & 100 \%             \\
        \textbf{001}      & Contratos de arrendamiento de vivienda entre particulares & 100 \%             \\
        \textbf{002}      & Contratos de compra venta de vehículo entre particulares  & 100 \%             \\
        \bottomrule
    \end{tabular}
    \caption{Precisión de los datos extraídos para cada conjunto}
    \label{tab:data_set_performance}
\end{table}