\section{Calidad del código y mantenibilidad}

La calidad del código y su mantenibilidad es elevada gracias al uso de los estándares de la industria descritos en las
secciones \ref{sec:clean_code} Código limpio y \ref{sec:clean_architecture} Arquitectura limpia.

Para evaluar la calidad del código, tendremos en cuenta los siguientes apartados.

\subsection*{Estándares de codificación}

En cuanto a los estándares de codificación, se han seguido los estándares definidos por el marco de trabajo de
\textbf{Symfony}~\cite{url_symfony_code_standards}.
Estos estándares aseguran que el código sea consistente a lo largo del tiempo.

Para verificar el cumplimiento de los estándares mencionados, se ha utilizado la herramienta \textbf{PHP Code Sniffer}.
Esta herramienta analiza el código fuente y verifica que no exista ningún incumplimiento de los estándares definidos.

\subsection*{Cobertura de Pruebas}

Para la ejecución de la batería de pruebas automáticas, hemos utilizado \textbf{PHPUnit}.
En total se han programado 36 pruebas, que realizan a su vez 109 comprobaciones diferentes.
La batería de pruebas se ejecuta en apenas 200 milisegundos en una instalación local.

\textbf{PHPUnit} permite generar un reporte detallado de la cobertura de código, indicando qué partes del código han
sido probadas y cuáles no.

En la tabla~\ref{tab:phpunit_report}, se muestra la cobertura de código obtenida para diferentes capas del sistema.

\begin{table}[h]
    \renewcommand{\arraystretch}{1.5}
    \setlength{\tabcolsep}{10pt}
    \begin{tabular}{>{\bfseries}p{0.75\textwidth} >{\raggedleft\arraybackslash}p{0.15\textwidth}}
        \toprule
        \textbf{Capa}                         & \textbf{Cobertura} \\
        \midrule
        \textbf{Dominio}                      & 72.41 \%           \\
        \textbf{Infraestructura y aplicación} & 42.86 \%           \\
        \bottomrule
    \end{tabular}
    \caption{Evaluación de la cobertura de código}
    \label{tab:phpunit_report}
\end{table}

Como puede verse, la capa de dominio tiene una cobertura del 72.41\% lo que se considera bastante buena.
La capa de dominio contiene la lógica central de negocio del sistema y es crucial que esté bien probada para asegurar
que el comportamiento del sistema sea correcto.

En cambio, una cobertura del 42.86\% en la capa de infraestructura y aplicación se considera moderada a baja.
Esta capa incluye componentes que interactúan con sistemas externos, bases de datos, \textit{API}, y otras
infraestructuras de soporte.
Esto se debe a que en ocasiones son capas más complicadas de testear, además de que hay situaciones que no merece la
pena incluir en la batería de pruebas debido a su complejidad o rareza.

Se recomienda mantener y mejorar la cobertura de código en ambas capas.

\subsection*{Análisis estático}

El análisis estático es el proceso de examinar el código fuente de un programa sin ejecutarlo para identificar posibles
errores, malas prácticas y problemas de seguridad.

Para este análisis, hemos utilizado la herramienta \textbf{Scrutinizer-CI}, la cual nos ha calificado con una puntuación
de diez sobre diez~\cite{url_scrutinizer_viu_84_proyecto}.

Esta herramienta analiza el código en busca de posibles problemas proporcionando un informe detallado así
sugerencias de mejora.

\subsection*{Complejidad del código}

Para evaluar la complejidad del código, hemos utilizado la herramienta \textbf{phploc} (\textit{PHP Lines of Code}), la
cual reporta varias métricas importantes.
A continuación, se presentan algunos de los resultados obtenidos, que aparecen en la tabla~\ref{tab:phploc_report}.

\begin{table}[h]
    \renewcommand{\arraystretch}{1.5}
    \setlength{\tabcolsep}{10pt}
    \begin{tabular}{p{0.60\textwidth} >{\raggedleft\arraybackslash}p{0.15\textwidth}
            >{\raggedleft\arraybackslash}p{0.15\textwidth}}

        \toprule
        \textbf{Nombre}                         & \textbf{Valor} & \textbf{Porcentaje} \\
        \midrule
        Total Non-Comment Lines of Code (NCLOC) & 1649           & 80.64 \%            \\
        Total Logical Lines of Code (LLOC)      & 313            & 15.31 \%            \\
        Average Class Length (LLOC)             & 7              &                     \\
        Average Complexity per Class            & 2.68           &                     \\
        Average Complexity per Method           & 1.38           &                     \\
        \bottomrule
    \end{tabular}
    \caption{Evaluación de la complejidad del código}
    \label{tab:phploc_report}
\end{table}

La métrica \textit{Total Non-Comment Lines of Code (NCLOC)} indica el número total de líneas de código que no son
comentarios.
Un mayor número de \textit{NCLOC} puede indicar un código más complejo o extenso, aunque no necesariamente más difícil
de mantener.

La métrica \textit{Total Logical Lines of Code (LLOC)} mide el número de líneas lógicas de código, excluyendo las
líneas vacías y los comentarios.
Las líneas lógicas representan instrucciones de programación, lo que da una idea más precisa de la cantidad de trabajo
realizado por el código.

\textit{Average Class Length (LLOC)} muestra la longitud promedio de las clases en términos de líneas lógicas de código.
Una menor longitud promedio de clase suele ser preferible, ya que indica clases más manejables y con responsabilidades
bien definidas.

\textit{Average Complexity per Class} mide la complejidad promedio de las clases utilizando la complejidad ciclomática,
que cuantifica el número de caminos independientes a través del código.
Una menor complejidad por clase sugiere un diseño más simple y fácil de entender.

Finalmente, \textit{Average Complexity per Method} similar a la métrica anterior, mide la complejidad ciclomática
promedio por método.
Métodos con menor complejidad son más fáciles de mantener y menos propensos a errores.

Estas métricas nos proporcionan una visión de un software bien estructurado y con baja complejidad.

\subsection*{Documentación}

El proyecto cuenta con una amplia documentación recogida en este TFG.
Esta documentación incluye tanto la descripción técnica del sistema como guías para desarrolladores y usuarios, lo que
facilita la comprensión y el uso del sistema.
La documentación cubre aspectos como la instalación, configuración, uso y mantenimiento del sistema, asegurando que
cualquier persona interesada pueda comprender y trabajar con el sistema de manera efectiva.

Esta documentación completa y detallada es esencial para garantizar la calidad y mantenibilidad del sistema a largo
plazo.

En resumen, las métricas analizadas y las herramientas utilizadas proporcionan una visión integral de un software bien
estructurado, con baja complejidad y alta mantenibilidad.
La adopción de estándares de la industria, como los sugeridos por Robert C. Martin en Código Limpio y Arquitectura
Limpia, aseguran que el código sea robusto.
La extensa documentación también contribuye a la comprensión y uso eficiente del sistema, garantizando su calidad y
mantenibilidad a largo plazo.












