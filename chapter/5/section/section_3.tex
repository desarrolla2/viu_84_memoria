\section{Calidad del código y mantenibilidad}

La calidad del código, y su mantenibilidad es elevada gracias al uso de los estándares de la industria descritos en
las secciones \ref{sec:clean_code} Código limpio y \ref{sec:clean_architecture} Arquitectura limpia.

Para evaluar la calidad del código tendremos en cuenta los siguientes apartados.

\subsection*{Estándares de codificación}

En cuanto a los estándares de codificación se han seguido los estandartes definidos por el marco de trabajo de
\textbf{Symfony}~\cite{https://symfony.com/doc/7.0/contributing/code/standards.html}.

Para verificar el cumplimiento de los estándares mencionados se ha utilizado la herramienta
\textbf{PHP\_CodeSniffer}~\cite{https://github.com/squizlabs/PHP_CodeSniffer}.
Con la cual se ha verificado que no existe ningún incumplimiento estándar de codificación definido.

\subsection*{Cobertura de Pruebas}

Para la ejecución de la batería de pruebas automáticas hemos utilizado \textbf{PHPUnit}.
Esta herramienta permite generar un reporte de cobertura de código.

Sin embargo, hemos calculado los tiempos de ejecución y el consumo de memoria para el documento 000/001.pdf con el
objetivo de establecer una referencia, tal y como se aprecia en la tabla~\ref{tab:phpunit_report}.

\begin{table}[h]
    \renewcommand{\arraystretch}{1.5}
    \setlength{\tabcolsep}{10pt}
    \centering
    \begin{tabular}{>{\bfseries}p{0.60\textwidth} p{0.25\textwidth}  }
        \toprule
        \textbf{Capa}                         & \textbf{Cobertura} \\
        \midrule
        \textbf{Dominio}                      & 72.41 \%           \\
        \textbf{Infraestructura y aplicación} & 42.86 \%           \\
        \bottomrule
    \end{tabular}
    \caption{Evaluación de la cobertura de código}
    \label{tab:phpunit_report}
\end{table}

Como puede verse la capa de dominio tiene una cobertura de código razonablemente buena, aunque aún tiene margen de
mejora.
En cambio, las capa de infraestructura y aplicación, tienen una menor cobertura, esto es debido a que en ocasiones son
capas más complicadas de testear, además que se dan situaciones que no merece la pena estar recogidas en la batería de
pruebas.

\subsection*{Análisis Estático}

Para el análisis estático del código hemos utilizado la herramienta \textbf{scrutinizer-ci}, la cual nos califica
con una puntuación de diez sobre diez~\cite{https://scrutinizer-ci.com/g/desarrolla2/viu_84_proyecto/}.

\subsection*{Complejidad del Código}

Para evaluar la complejidad del código hemos utilizado la herramienta \textbf{phploc}, la cual reporta entre otros los
datos que aparecen en la tabla~\ref{tab:phploc_report}

\begin{table}[h]
    \renewcommand{\arraystretch}{1.5}
    \setlength{\tabcolsep}{10pt}
    \centering
    \begin{tabular}{>{\bfseries}p{0.60\textwidth} p{0.25\textwidth}  }
        \toprule
        \textbf{Nombre}                         & \textbf{Valor} \\
        \midrule
        Total Non-Comment Lines of Code (NCLOC) & 1649 (80.64\%) \\
        Total Logical Lines of Code (LLOC)      & 313 (15.31\%)  \\
        Average Class Length (LLOC)             & 7              \\
        Average Complexity per Class            & 2.68           \\
        Average Complexity per Method           & 1.38           \\
        \bottomrule
    \end{tabular}
    \caption{Evaluación de la complejidad del código}
    \label{tab:phploc_report}
\end{table}

\subsection*{Documentación}

El proyecto cuenta con una amplia documentación recogida en este TFG.