\section{Descripción detallada de la solución informática desarrollada}

Se ha desarrollado un sistema capaz de extraer información de documentos.
Para el desarrollo se ha seguido un conjunto de buenas prácticas cómo por ejemplo el uso de arquitecturas límpias,
esto facilita que la solución pueda ser extendida, para diferentes casos de uso.

El sistema tiene un primer componente, el componente \textit{Generator} con capacidad para convertir documentos
\textbf{PDF} en texto, tal y como definimos en el \textbf{Requisito 1}.

Aunque en esta implementación se han utilizado un sistema concreto para la conversión de documentos \textbf{PDF}
en texto, el sistema permite incorporar otras implementaciones, que hagan la conversión utilizando otros elementos de
infraestructura, o añadiendo soporte para otros formatos.

El sistema tiene un segundo componente, el componente \textit{Reader} capaz de extraer información a partir de la
salida del componente anterior.
Se han implementado el soporte para dos tipos de documento concretos que son los contratos de arrendamiento de vivienda
entre particulares \textbf{Requisito 2} y contratos de compraventa de vehículo entre particulares \textbf{Requisito 3}.

De la misma forma, que con el componente anterior, es sencillo añadir soporte para nuevos tipos de documentos.



\subsection*{Interfaces de Usuario}
Para interactuar con el sistema, se han desarrollado dos interfaces:

\begin{itemize}
    \item Interfaz de Línea de Comandos: Esta herramienta está dirigida a desarrolladores y
    administradores del sistema, permitiendo ejecutar comandos y scripts directamente.


    \item Interfaz Web: Presenta un área donde los usuarios pueden
    arrastrar y soltar documentos para su análisis y muestra una representación en formato JSON de la información
    extraída.
\end{itemize}


\subsection*{Interfaces de usuario}
Aunque el objetivo principal de este trabajo es desarrollar una tecnología, flexible, extensible y que pueda ser
fácilmente integrada en otros sistemas, se han implementado dos tipos de interfaces sencillas, que permiten demostrar el
correcto funcionamiento de la tecnología.

\subsection*{Interfaz de línea de comandos}
La interfaz de línea de comandos es una herramienta para desarrolladores y administradores del sistema. Permite ejecutar
comandos y scripts de manera directa, facilitando la automatización de tareas y la integración con otros sistemas.


Ejecución de la aplicación a través de la línea de comandos

Esta interfaz recibe como parámetro la ruta de un fichero que se pretende analizar y una vez analizado muestra una
representación en formato tabla de la información extraída del mismo.

\subsection*{Interfaz web}
La interfaz web es el principal punto de interacción para la mayoría de los usuarios. Está diseñada para ser intuitiva,
accesible y eficiente, permitiendo a los usuarios realizar una amplia gama de operaciones a través de un navegador web.

Para este proyecto hemos desarrollado una interfaz con las siguientes características

\begin{itemize}
    \item
    Diseño Responsive: La interfaz web está diseñada para ser accesible desde dispositivos de escritorio y móviles,
    asegurando una experiencia de usuario coherente y optimizada en diferentes tamaños de pantalla.
    \item
    Experiencia de usuario intuitiva: Se ha prestado atención a la usabilidad, con una interfaz limpias y fáciles de
    utilizar y retroalimentación inmediata a las acciones del usuario.
\end{itemize}


Ejecución de la aplicación a través de la interfaz web.

Esta interfaz muestra un área sobre la que se pueden arrastrar y soltar documentos, una vez recibidos, y analizados
muestra una representación en formato json de la información extraída del mismo.

\colorbox{color_highlight} {@TODO: }
