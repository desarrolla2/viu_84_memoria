\chapter{Resultados}\label{ch:chapter_5}


\section{Pruebas y análisis de los resultados}
\subsection*{Interfaces de Usuario}
Para interactuar con el sistema, se han desarrollado dos interfaces:

\begin{itemize}
    \item Interfaz de Línea de Comandos: Esta herramienta está dirigida a desarrolladores y
    administradores del sistema, permitiendo ejecutar comandos y scripts directamente.


    \item Interfaz Web: Presenta un área donde los usuarios pueden
    arrastrar y soltar documentos para su análisis y muestra una representación en formato JSON de la información
    extraída.
\end{itemize}


\subsection*{Interfaces de usuario}
Aunque el objetivo principal de este trabajo es desarrollar una tecnología, flexible, extensible y que pueda ser
fácilmente integrada en otros sistemas, se han implementado dos tipos de interfaces sencillas, que permiten demostrar el
correcto funcionamiento de la tecnología.

\subsection*{Interfaz de línea de comandos}
La interfaz de línea de comandos es una herramienta para desarrolladores y administradores del sistema. Permite ejecutar
comandos y scripts de manera directa, facilitando la automatización de tareas y la integración con otros sistemas.


Ejecución de la aplicación a través de la línea de comandos

Esta interfaz recibe como parámetro la ruta de un fichero que se pretende analizar y una vez analizado muestra una
representación en formato tabla de la información extraída del mismo.

\subsection*{Interfaz web}
La interfaz web es el principal punto de interacción para la mayoría de los usuarios. Está diseñada para ser intuitiva,
accesible y eficiente, permitiendo a los usuarios realizar una amplia gama de operaciones a través de un navegador web.

Para este proyecto hemos desarrollado una interfaz con las siguientes características

\begin{itemize}
    \item
    Diseño Responsive: La interfaz web está diseñada para ser accesible desde dispositivos de escritorio y móviles,
    asegurando una experiencia de usuario coherente y optimizada en diferentes tamaños de pantalla.
    \item
    Experiencia de usuario intuitiva: Se ha prestado atención a la usabilidad, con una interfaz limpias y fáciles de
    utilizar y retroalimentación inmediata a las acciones del usuario.
\end{itemize}


Ejecución de la aplicación a través de la interfaz web.

Esta interfaz muestra un área sobre la que se pueden arrastrar y soltar documentos, una vez recibidos, y analizados
muestra una representación en formato json de la información extraída del mismo.

\colorbox{color_highlight} {@TODO: }
explicar de nuevo que seha desarrollado una test suite de pruebas unitarias, y
completar indicando el porcentaje de acierto para las pruebas manuales.

Resultados cuantitativos: Presenta los resultados numéricos o medibles de tu
proyecto.
Esto puede incluir métricas de rendimiento, tiempos de respuesta, velocidad de procesamiento, eficiencia,
precisión, entre otros.
Utiliza tablas, gráficos u otros medios visuales para mostrar claramente los datos recopilados.

Resultados cualitativos: Si tu proyecto implica evaluaciones subjetivas o cualitativas, como la usabilidad, la
experiencia del usuario o la calidad percibida,describe los resultados obtenidos a través de encuestas, entrevistas o
pruebas de usabilidad.

Pruebas y resultados funcionales: se presentan los resultados de las pruebas realizadas para verificar el correcto
funcionamiento de la solución.

Muestra cómo se han llevado a cabo las pruebas y los casos de prueba utilizados.
Destaca los resultados obtenidos en términos de la funcionalidad y el cumplimiento de los requisitos establecidos.

Casos de estudio o resultados específicos: Si has realizado estudios de casos específicos o evaluaciones particulares,
describe los resultados obtenidos y su relevancia para tu proyecto.
Puedes incluir ejemplos concretos de cómo la solución informática ha sido aplicada en situaciones reales y los
resultados obtenidos en cada caso.

Validación y pruebas: Explica cómo se han validado y evaluado los resultados de tu proyecto.
Si has realizado pruebas, verifica que se cumplan los requisitos establecidos y describe cómo se ha llevado a cabo la
evaluación.
Si se han utilizado conjuntos de datos de prueba o casos de uso específicos, menciónalos en esta sección.

Comparación con resultados esperados: Compara tus resultados con los objetivos
y las expectativas establecidos en la introducción de tu trabajo.
Destaca si has logrado alcanzar tus metas y si los resultados obtenidos son consistentes con las hipótesis o
predicciones iniciales.
Si hay desviaciones o discrepancias, explícalas y proporciona posibles explicaciones.

Análisis de los resultados: Realiza un análisis de los resultados obtenidos y su relevancia para tu proyecto.


\section{Rendimiento}

// @TODO: Rendimiento y eficiencia: evaluación del rendimiento y la eficiencia de la solución informática.
Muestra los resultados obtenidos en términos de tiempos de respuesta,velocidad de procesamiento, uso de recursos,
escalabilidad, entre otros aspectos relevantes para tu proyecto.
Compara los resultados con los objetivos establecidos.


\section{Calidad del código}

// @TODO, Calidad del código y mantenibilidad:
Evalúa la calidad del código desarrollado y la mantenibilidad de la solución.


\section{Limitaciones}

// @TODO, Limitaciones y posibles mejoras: Menciona las limitaciones o restricciones
que puedan haber afectado sus resultados.
Esto puede incluir limitaciones en los datos, en los métodos utilizados o en la implementación del proyecto.
También puedes sugerir posibles mejoras o áreas de investigación futuras basadas en las limitaciones identificadas.


\section{Coste}
// @TODO,


\section{Posibles mejoras}

\subsection*{OCR}
\colorbox{color_highlight}{@TODO:}

\subsection*{Nuevos Readers}
// @TODO:
