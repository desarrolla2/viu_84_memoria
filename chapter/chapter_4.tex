\chapter{Resultados}\label{ch:chapter_4}


\section{Descripción de la solución}
El sistema desarrollado consta de una arquitectura basada en dos componentes principales: Generator y Reader.

Ambos componentes están diseñados para ser fácilmente extensibles, mediante un sistema en el que se pueden registrar
nuevos pre procesadores, procesadores y post procesadores, para cumplir con los requisitos de casos de uso específicos.

\subsection{Componente Generator}
El componente Generator se encarga de convertir documentos en diferentes formatos a texto plano.
En esta implementación, se desarrolló un procesador especializado que utiliza la herramienta pdftotext para transformar
documentos PDF en texto.

La arquitectura modular del Generator permite la inclusión de nuevos pre procesadores y post procesadores para ajustar y
perfeccionar el texto generado.

\subsection{Componente Generator}
El componente Reader interpreta y extrae la información estructurada del texto generado por el Generator.

Funciona mediante un sistema de procesadores organizados en una única capa, que opera bajo un mecanismo competitivo
similar al del componente Generator.
Los procesadores compiten entre sí para determinar cuál es el más adecuado para analizar y extraer la información
necesaria del texto.

En esta implementación, se desarrollaron dos procesadores que consumen la api pública de OpenAI, para extraer la
información de dos tipos de contratos diferentes: contratos de arrendamiento entre particulares y contratos de
compraventa de vehículos entre particulares.

\subsection{Interfaces de Usuario}
Para interactuar con el sistema, se han desarrollado dos interfaces:

\begin{itemize}
    \item Interfaz de Línea de Comandos: Esta herramienta está dirigida a desarrolladores y
    administradores del sistema, permitiendo ejecutar comandos y scripts directamente.


    \item Interfaz Web: Presenta un área donde los usuarios pueden
    arrastrar y soltar documentos para su análisis y muestra una representación en formato JSON de la información
    extraída.
\end{itemize}


\section{Pruebas y análisis de los resultados}

\colorbox{color_highlight} {@TODO: }
explicar de nuevo que seha desarrollado una test suite de pruebas unitarias, y
completar indicando el porcentaje de acierto para las pruebas manuales.

Resultados cuantitativos: Presenta los resultados numéricos o medibles de tu
proyecto.
Esto puede incluir métricas de rendimiento, tiempos de respuesta, velocidad de procesamiento, eficiencia,
precisión, entre otros.
Utiliza tablas, gráficos u otros medios visuales para mostrar claramente los datos recopilados.

Resultados cualitativos: Si tu proyecto implica evaluaciones subjetivas o cualitativas, como la usabilidad, la
experiencia del usuario o la calidad percibida,describe los resultados obtenidos a través de encuestas, entrevistas o
pruebas de usabilidad.

Pruebas y resultados funcionales: se presentan los resultados de las pruebas realizadas para verificar el correcto
funcionamiento de la solución.

Muestra cómo se han llevado a cabo las pruebas y los casos de prueba utilizados.
Destaca los resultados obtenidos en términos de la funcionalidad y el cumplimiento de los requisitos establecidos.

Casos de estudio o resultados específicos: Si has realizado estudios de casos específicos o evaluaciones particulares,
describe los resultados obtenidos y su relevancia para tu proyecto.
Puedes incluir ejemplos concretos de cómo la solución informática ha sido aplicada en situaciones reales y los
resultados obtenidos en cada caso.

Validación y pruebas: Explica cómo se han validado y evaluado los resultados de tu proyecto.
Si has realizado pruebas, verifica que se cumplan los requisitos establecidos y describe cómo se ha llevado a cabo la
evaluación.
Si se han utilizado conjuntos de datos de prueba o casos de uso específicos, menciónalos en esta sección.

Comparación con resultados esperados: Compara tus resultados con los objetivos
y las expectativas establecidos en la introducción de tu trabajo.
Destaca si has logrado alcanzar tus metas y si los resultados obtenidos son consistentes con las hipótesis o
predicciones iniciales.
Si hay desviaciones o discrepancias, explícalas y proporciona posibles explicaciones.

Análisis de los resultados: Realiza un análisis de los resultados obtenidos y su relevancia para tu proyecto.


\section{Rendimiento}

// @TODO: Rendimiento y eficiencia: evaluación del rendimiento y la eficiencia de la solución informática.
Muestra los resultados obtenidos en términos de tiempos de respuesta,velocidad de procesamiento, uso de recursos,
escalabilidad, entre otros aspectos relevantes para tu proyecto.
Compara los resultados con los objetivos establecidos.


\section{Calidad del código}

// @TODO, Calidad del código y mantenibilidad:
Evalúa la calidad del código desarrollado y la mantenibilidad de la solución.


\section{Limitaciones}

// @TODO, Limitaciones y posibles mejoras: Menciona las limitaciones o restricciones
que puedan haber afectado sus resultados.
Esto puede incluir limitaciones en los datos, en los métodos utilizados o en la implementación del proyecto.
También puedes sugerir posibles mejoras o áreas de investigación futuras basadas en las limitaciones identificadas.


\section{Coste}
// @TODO,


\section{Posibles mejoras}

\subsection{OCR}
\colorbox{color_highlight}{@TODO:}

\subsection{Nuevos Readers}
// @TODO:
