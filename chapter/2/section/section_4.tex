\section{Modelos de lenguaje de gran escala}

La historia de los Modelos de lenguaje de gran escala o \textit{Large Language Models} (LLMs) comienza con los
primeros modelos estadísticos de lenguaje en la década de 1980, que utilizaban métodos simples como los modelos de
Markov y n-gramas para predecir la probabilidad de secuencias de palabras~\cite{article_jelinek_1997}.

Estos métodos, aunque efectivos para algunas tareas básicas, estaban limitados por su incapacidad para capturar
contextos más largos y por su dependencia de grandes corpus de texto para entrenamiento.

En la década de 2000, con el advenimiento de modelos más sofisticados como los modelos ocultos de Markov y especialmente
las redes neuronales, comenzó a vislumbrarse el potencial de los modelos de lenguaje más complejos.

Sin embargo, no fue hasta la introducción de las redes neuronales recurrentes o \textit{recurrent neural networks} (RNN)
y, más tarde, las redes neuronales de memoria a largo plazo o \textit{long-term memory neural networks} (LSTM) que
los investigadores pudieron abordar el problema del ``desvanecimiento del gradiente'' y mejorar significativamente la
capacidad de los modelos para aprender dependencias a largo plazo en el texto~\cite{article_hochreiter_1997}.

El verdadero cambio paradigmático llegó en 2017 con el desarrollo de la arquitectura \textit{Transformer}
~\cite{article_vaswani_2017}.
Esta arquitectura introdujo el mecanismo de atención, que permite a los modelos ponderar diferentes partes de la entrada
de texto de manera dinámica, mejorando la capacidad de los modelos para manejar secuencias de texto largas y complejas.

En este proyecto se utilizaron modelos de lenguaje de gran escala como \textit{GPT}, para la extracción de información
de documentos, tal y cómo veremos en la sección~\ref{sec:implemetacion_y_programacion} Implementación y programación.

\subsection*{Principales modelos}

\subsubsection*{GPT}
GPT o \textit{Generative Pre-trained Transformer} es un avanzado modelo de lenguaje desarrollado por OpenAI
~\cite{article_brown_2020}.
Es uno de los modelos más reconocidos en la actualidad.
Aunque es un modelo privado, \textit{OpenAI} ofrece acceso a través de una \textit{API} de pago que incluye una capa
gratuita limitada.
\textit{GPT} ofrece varios modelos, cada uno con un esquema de precios que varía según su complejidad.
Una de las principales ventajas de este servicio es que proporciona una solución integral \textit{plug-and-play}, es
decir, no requiere instalación ni configuración adicional por parte del usuario.

\subsubsection*{BERT}
\textit{BERT}, desarrollado por \textit{Google}~\cite{article_devlin_2019}, es un modelo de lenguaje basado en la
arquitectura \textit{Transformer} que se entrenó utilizando un gran corpus de texto en una manera bidireccional, lo
que le permite entender el contexto de una palabra basada en todas sus apariciones en un texto.

\subsubsection*{Hugging Face's Transformers}

\textit{Hugging Face's Transformers} es una biblioteca de NLP~\cite{article_wolf_2020} que simplifica el uso de modelos
de aprendizaje profundo basados en la arquitectura \textit{Transformer}.

Esta biblioteca ofrece una amplia variedad de modelos pre entrenados y libres desarrollados por la comunidad para fines
específicos, además de la posibilidad de entrenar modelos personalizados.

En contraposición, los modelos disponibles en \textit{Hugging Face}
suelen ser menos potentes que los modelos más disponibles a través de plataformas como \textit{GPT} de
\textit{OpenAI}~\cite{url_chatgpt_vs_huggingchat}.

\subsubsection*{Otros}

Además de los descritos existen muchas otras plataformas o proyectos como por ejemplo
\textit{Eleuther AI}~\cite{url_eleutherai}, \textit{Fairseq}~\cite{url_fairseq} o \textit{GPT 2}~\cite{url_openai_gpt2}
que ofrecen alternativas LLM para desarrolladores interesados en incorporar este tipo de capacidades a sus aplicaciones.
