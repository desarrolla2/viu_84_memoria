\section{Código limpio}\label{sec:clean_code}

En el desarrollo del sistema de extracción de información de documentos, permitirá que otros desarrolladores pudieran
entender y modificar el código con facilidad, lo cual es esencial para futuras extensiones y mejoras del sistema.

\subsection*{Introducción histórica}
El concepto de Código Limpio ha sido fundamental en la programación de software desde los primeros días del desarrollo
de software, pero fue articulado y popularizado con gran efecto por Robert C. Martin en su libro~\cite{book_martin_2008}

Este libro se convirtió en una guía esencial para muchos desarrolladores al enfatizar la importancia de escribir código
que no solo funcione, sino que también sea fácil de entender, modificar y mantener.

\subsection*{Definición}
El código limpio es aquel que es fácil de entender y fácil de modificar y hace exactamente lo que se espera que haga.
Según Robert C. Martin, el código limpio puede ser leído y mejorado por un desarrollador que no sea su autor original
con un mínimo esfuerzo necesario.

Se caracteriza por su simplicidad, la ausencia de duplicación, la expresión clara de la intención del desarrollador y
la atención a los detalles en el nivel de código.
