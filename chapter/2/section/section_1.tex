\section{Código limpio}\label{sec:clean_code}

El concepto de Código Limpio ha sido fundamental en la programación de software desde los primeros días del desarrollo
de software, pero fue articulado y popularizado por Robert C. Martin en el libro
\textit{Clean Code: A Handbook of Agile Software Craftsmanship}~\cite{book_martin_2008}

Este libro se convirtió en una guía esencial para muchos desarrolladores al enfatizar la importancia de escribir código
que no solo funcione, sino que también sea fácil de entender, modificar y mantener.

El código limpio es aquel que es fácil de entender y fácil de modificar y hace exactamente lo que se espera que haga.
Según Robert C. Martin, el código limpio puede ser leído y mejorado por un desarrollador que no sea su autor original
con un mínimo esfuerzo necesario.

Se caracteriza por su simplicidad, la ausencia de duplicación, la expresión clara de la intención del desarrollador y
la atención a los detalles en el nivel de código.

En el desarrollo de este proyecto, permitirá que otros desarrolladores puedan entender y modificar el código con
facilidad, lo cual es esencial para futuras extensiones y mejoras del sistema.
