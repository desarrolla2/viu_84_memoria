\section{Arquitectura del sistema}


\section{Interfaces de usuario}
Aunque el objetivo principal de este trabajo es desarrollar una tecnología, flexible, extensible y que pueda ser
fácilmente integrada en otros sistemas, se han implementado dos tipos de interfaces sencillas, que permiten demostrar el
correcto funcionamiento de la tecnología.

\subsection*{Interfaz de línea de comandos}
La interfaz de línea de comandos es una herramienta para desarrolladores y administradores del sistema. Permite ejecutar
comandos y scripts de manera directa, facilitando la automatización de tareas y la integración con otros sistemas.


Ejecución de la aplicación a través de la línea de comandos

Esta interfaz recibe como parámetro la ruta de un fichero que se pretende analizar y una vez analizado muestra una
representación en formato tabla de la información extraída del mismo.

\subsection*{Interfaz web}
La interfaz web es el principal punto de interacción para la mayoría de los usuarios. Está diseñada para ser intuitiva,
accesible y eficiente, permitiendo a los usuarios realizar una amplia gama de operaciones a través de un navegador web.

Para este proyecto hemos desarrollado una interfaz con las siguientes características

\begin{itemize}
    \item
    Diseño Responsive: La interfaz web está diseñada para ser accesible desde dispositivos de escritorio y móviles,
    asegurando una experiencia de usuario coherente y optimizada en diferentes tamaños de pantalla.
    \item
    Experiencia de usuario intuitiva: Se ha prestado atención a la usabilidad, con una interfaz limpias y fáciles de
    utilizar y retroalimentación inmediata a las acciones del usuario.
\end{itemize}


Ejecución de la aplicación a través de la interfaz web.

Esta interfaz muestra un área sobre la que se pueden arrastrar y soltar documentos, una vez recibidos, y analizados
muestra una representación en formato json de la información extraída del mismo.

\section{Caché}
La implementación de una caché en aplicaciones web es una técnica comúnmente utilizada para mejorar el rendimiento y la
eficiencia..

En este proyecto, la caché se ha utilizado específicamente para almacenar las peticiones HTTP realizadas, proporcionando
varias ventajas significativas, como el aumento de la velocidad de respuesta y la reducción de costos operativos..

Aunque en un entorno de producción la utilidad de esta caché podría ser limitada, en un entorno de desarrollo resulta
invaluable. Esto se debe a que en desarrollo se trabaja principalmente con un conjunto de datos más pequeño y las
peticiones se repiten con frecuencia.

\section{Registro y gestión de logs}
El registro de logs es una parte crucial del monitoreo y mantenimiento de cualquier aplicación..

En este proyecto, se ha implementado un sistema de logging utilizando Monolog, una biblioteca de registro para PHP.

Además de los logs que almacena symfony por defecto, se han configurado 3 canales adicionales:

\begin{itemize}
    \item Generator: para los logs del componente generator.
    \item Reader: para los logs del componente reader.
    \item Http-Client: para el componente que realiza las peticiones HTTP.
\end{itemize}

Un canal es la forma en la que monolog, agrupa un conjunto de información para poderla filtrar y procesar adecuadamente.

Canal además se registra en dos ficheros diferentes

\begin{itemize}
    \item Log File format, es el formato estándar de ficheros de logs.
    \item Logstash format, es el estándar de la herramienta logstash.
\end{itemize}

\subsection*{Formato de ficheros por defecto}
El formato por defecto es adecuado para entornos de desarrollo o proyectos de pequeña envergadura. Siempre que estos
ficheros no sean demasiado grandes, se pueden trabajar a través de herramientas de línea de comandos como:

\begin{itemize}
    \item grep: Utilizado para buscar patrones específicos dentro de los archivos de logs.
    \item awk: Utilizado para procesar y analizar los logs de manera más compleja.
\end{itemize}

\subsection*{Sistema ELK}
Un sistema compuesto por Elasticsearch, Logstash y Kibana (ELK) es una solución centralizada de gestión de logs, que
permite una monitorización más avanzada de los mismos. Está indicado en entornos de producción, donde la monitorización
de logs sea una tarea importante..


Monitorización de logs a través de un sistema ELK

El sistema ELK se compone de tres componentes:

\begin{itemize}
    \item
    Logstash: Es una herramienta de procesamiento de datos que ingiere, transforma y envía datos a diversos destinos,
    siendo Elasticsearch uno de los más comunes.
    \item
    Elasticsearch: Es un motor de búsqueda y análisis de texto completo basado en Lucene. Permite almacenar, buscar y
    analizar grandes volúmenes de datos en tiempo real.
    \item Kibana: Es una herramienta de visualización de datos que trabaja en conjunto con Elasticsearch. Permite a los
    usuarios crear gráficos y dashboards interactivos para visualizar y analizar los datos de logs almacenados en
    Elasticsearch.
\end{itemize}

\section{Contenedores}
La contenedorización es una técnica que permite encapsular una aplicación y sus dependencias en uno o más contenedores,
lo que garantiza que se ejecutará de manera consistente en cualquier entorno..

\subsection*{Docker}
Docker permite empaquetar la aplicación junto con todas sus dependencias en una imagen de contenedor. Esta imagen puede
ser ejecutada en cualquier máquina que tenga Docker instalado, asegurando consistencia y eliminando problemas
relacionados con diferencias en el entorno de desarrollo y producción.

En este proyecto, se ha utilizado Docker para contenerizar la aplicación, proporcionando un entorno de desarrollo y
despliegue robusto y reproducible.

\subsection*{Docker compose}
Docker Compose se utiliza para definir y ejecutar aplicaciones Docker multi-contenedor. En este proyecto, Docker Compose
gestiona la aplicación PHP junto con otros servicios necesarios, como el conjunto ELK
