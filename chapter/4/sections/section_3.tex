\section{Desarrollo de software}\label{sec:Desarrollo de software}


\subsection*{Interfaces de usuario}
Aunque el objetivo principal de este trabajo es desarrollar una tecnología, flexible, extensible y que pueda ser
fácilmente integrada en otros sistemas, se han implementado dos tipos de interfaces sencillas, que permiten demostrar el
correcto funcionamiento de la tecnología.

\subsection*{Interfaz de línea de comandos}
La interfaz de línea de comandos es una herramienta para desarrolladores y administradores del sistema. Permite ejecutar
comandos y scripts de manera directa, facilitando la automatización de tareas y la integración con otros sistemas.


Ejecución de la aplicación a través de la línea de comandos

Esta interfaz recibe como parámetro la ruta de un fichero que se pretende analizar y una vez analizado muestra una
representación en formato tabla de la información extraída del mismo.

\subsection*{Interfaz web}
La interfaz web es el principal punto de interacción para la mayoría de los usuarios. Está diseñada para ser intuitiva,
accesible y eficiente, permitiendo a los usuarios realizar una amplia gama de operaciones a través de un navegador web.

Para este proyecto hemos desarrollado una interfaz con las siguientes características

\begin{itemize}
    \item
    Diseño Responsive: La interfaz web está diseñada para ser accesible desde dispositivos de escritorio y móviles,
    asegurando una experiencia de usuario coherente y optimizada en diferentes tamaños de pantalla.
    \item
    Experiencia de usuario intuitiva: Se ha prestado atención a la usabilidad, con una interfaz limpias y fáciles de
    utilizar y retroalimentación inmediata a las acciones del usuario.
\end{itemize}


Ejecución de la aplicación a través de la interfaz web.

Esta interfaz muestra un área sobre la que se pueden arrastrar y soltar documentos, una vez recibidos, y analizados
muestra una representación en formato json de la información extraída del mismo.
