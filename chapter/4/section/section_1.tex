\section{Planificación}\label{sec:desarrollo_de_software}

Este proyecto parte de una base realizada en la asignatura 47 GIIN Proyecto de ingeniería del
software de este mismo grado universitario~\cite{url_viu_47_proyecto}.

\subsection*{Objetivos}

A partir del trabajo realizado en esa asignatura se plantean los siguientes objetivos:

\begin{enumerate}
    \item \textbf{Definir un nuevo conjunto de datos de prueba},
    En el trabajo previo existía un conjunto de datos de prueba.
    Sin embargo, para este TFG se quería definir un nuevo conjunto, utilizando si fuera posible algún conjunto existente
    de datos de fuentes abiertas.
    \item \textbf{Evaluar la precisión del sistema},
    se quería evaluar cómo de preciso es el sistema a la hora de la extracción de información de los documentos.
    \item \textbf{Experimentar y seleccionar uno o más sistemas LLM}
    en el trabajo previo se había probado con los modelos \textit{ollama}~\cite{url_ollama} y
    \textit{gpt-3.5-turbo}~\cite{url_openai_gpt3_5}.
    Para este TFG se quería probar con nuevos modelos.
    \item \textbf{Mejora de las interfaces}
    se querían realizar mejoras en la capa de presentación de ambas interfaces, en el sentido de hacerlas más
    visuales, concretamente en lo relacionado con la representación de los datos.
    \item \textbf{Documentación formal},
    será necesaria la elaboración de la documentación requerida para completar satisfactoriamente este TFG.
\end{enumerate}

Debían respetarse además los tres requisitos definidos previamente en el capítulo~\ref{ch:chapter_1} Introducción.

\subsection*{Plan de trabajo}

Una vez que se han identificado los requisitos y los objetivos se estableció un plan de trabajo, que además tal y como
se describió en la sección~\ref{subsec:iterativo_incremental} Iterativo incremental, al final de cada sprint se realizó
una pequeña retrospectiva, para evaluar que tan bien estaba funcionando el plan de trabajo, así como para planificar
la siguiente iteración y modificar si fuera necesario el plan de trabajo restante en función de los resultados obtenidos
hasta ese momento.

\subsubsection*{Iteración 0: Hasta el 24 de marzo}

Se realizó el planteamiento inicial del TFG, la elaboración y entrega del anteproyecto.
Se realizó la instalación del proyecto y una nueva selección de las herramientas de trabajo.

\subsubsection*{Iteración 1: del 25 de marzo al 7 de abril}

Se realizó una ampliación del conjunto de datos de prueba.
Para ello se realizó una investigación sobre distintas bases de datos que ofrecen conjuntos de datos abiertos; sin
embargo, estas fuentes de datos ofrecen conjuntos de datos anonimizados, es decir datos como nombres, apellidos,
números de documentos, direcciones, no aparecen.

Es por este motivo que se descartó utilizar uno de estos conjuntos de datos.

\subsubsection*{Iteración 2: del 8 de abril al 21 de abril}

Al final de la iteración anterior se determinó que el conjunto de datos de prueba debería realizarse manualmente.
Para ello se desarrolló un sistema que permitiría, escribir los contratos en formato \textit{Markdown} el cual es fácil
de manipular y luego sería convertido a \textit{PDF} utilizando \textit{LaTeX} y diferentes plantillas para introducir
variabilidad en los formatos de entrada.

\subsubsection*{Iteración 3: del 22 de abril al 5 de mayo}

En apenas unos meses desde el final de la asignatura en la que se basa este trabajo, han aparecido una serie de nuevos
modelos.
Se hacía necesario investigar el funcionamiento de modelos como \textit{Llama 3}~\cite{url_llama3} y
\textit{ChatGPT 4}~\cite{url_openai_gpt4} para determinar si eran más adecuados que los modelos que se utilizaron en
dicho trabajo.

Después de realizar una serie de pruebas durante esta iteración, se determinó que si bien \textit{Llama 3} es un
modelo mucho más avanzado con respecto a su versión anterior, \textit{ChatGPT 4} demostró ser muy superior, por lo que
se determinó que se realizaría una nueva implementación del proyecto utilizando este modelo.

\subsubsection*{Iteración 4: del 6 de mayo al 19 de mayo}

Esta iteración estuvo enfocada en realizar una reimplementación de los procesadores
\textit{Residential Lease Agreement Processor} y \textit{Vehicle Sale And Purchase Agreement Processor}
que desarrollaremos en la sección~\ref{sec:diseno_del_sistema}.

\subsubsection*{Iteración 5: del 20 de mayo al 2 de junio}

Si bien durante todo el proyecto se han desarrollado las pruebas unitarias, de integración y funcionales oportunas, esta
iteración se dedicó a realizar una revisión formal de cómo de bueno era la implementación realizada en comparación
con los datos reales que contiene el conjunto de datos de prueba desarrollado.

En esta iteración, se realizó una primera versión del borrador del TFG. Para ello se utilizó la herramienta web
\textit{Google Docs}~\cite{url_google_docs}.
Si bien los resultados con esta herramienta fueron muy buenos en cuanto a productividad y acabado de los documentos
generados, no cumplía con algunos de los requisitos necesarios para la realización del TFG.

\subsubsection*{Iteración 6: del 3 de junio al 16 de junio}

En este segunda iteración dedicado a la elaboración de la memoria fue necesario realizar una nueva versión de este
documento, pero esta vez utilizando \textit{LaTeX}~\cite{url_latex}, una herramienta más compleja de utilizar pero
mucho más adecuada a la hora de realizar este tipo de trabajos.

\subsubsection*{Iteración 7: del 17 de junio al 30 de junio}

La última iteración se dedicó a realizar los ajustes finales y revisión de la memoria, y comenzar a preparar la
presentación.
