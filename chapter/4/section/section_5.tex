\section{Pruebas y evaluación}\label{sec:pruebas_y_evaluacion}

Detalla las pruebas realizadas para verificar el correcto funcionamiento de tu solución.
Explica los casos de prueba utilizados, los resultados obtenidos y cómo se han abordado los problemas o errores
identificados.
Si has realizado pruebas de rendimiento, seguridad u otros aspectos, también inclúyelos en esta sección.


\subsection{Pruebas}
Las pruebas en este proyecto son fundamentales para asegurar que la tecnología funciona conforme a las expectativas y
requisitos establecidos. Para garantizar una validación exhaustiva y precisa, se ha diseñado cuidadosamente un sistema
de pruebas descrito a continuación.

\subsection*{Construcción del conjunto de datos de prueba}

\colorbox{color_highlight}{@TODO: @marlene:} menciona el tamaño del dataset
El proceso para encontrar un conjunto de datos de prueba fue complejo, ya que no se encontraron conjuntos de datos de
fuentes abiertas que cumplieran con los requisitos necesarios..

La mayoría de estos conjuntos se distribuyen anonimizados, y nuestra tecnología necesita identificar y extraer
principalmente datos de carácter personal, como nombres, documentos de identidad, direcciones, fechas, importes, números
de cuentas bancarias, entre otros.

Finalmente, se construyó manualmente un conjunto de datos de prueba utilizando modelos de documentos encontrados en
internet, así como contratos reales a los que tuvimos acceso..

Este conjunto de datos se compone de tres subconjuntos:


INSERTAR TABLA

El siguiente paso fue modificar todos los documentos para que todos los datos de carácter personal fueran aleatorios y
no correspondiera a personas reales. Este proceso aseguró la privacidad y cumplió con las normativa de protección de
datos.

Para facilitar la modificación y manejo de estos documentos, se guardaron en formato Markdown, lo que permitió una
edición más rápida en el propio editor de código.

Dado que la entrada del sistema acepta documentos en formato PDF, se desarrolló un pequeño script para convertir cada
documento Markdown en PDF. Este script utiliza plantillas LaTeX para introducir variabilidad en el formato de los
documentos, asegurando una mayor robustez en las pruebas del sistema.


Este enfoque en la creación del conjunto de datos de prueba garantiza que el sistema pueda ser evaluado de manera
exhaustiva y precisa, cubriendo diversos escenarios y tipos de documentos relevantes para la tecnología desarrollada.

\subsection*{Pruebas automáticas}
Las pruebas automáticas son esenciales para garantizar la calidad y la robustez del software. Permiten verificar que el
código se comporta como se espera, identificar errores de manera temprana y asegurar que las nuevas funcionalidades no
introduzcan fallos en el sistema existente..

En este proyecto, se han implementado varios tipos de pruebas automáticas, cada una con un enfoque diferente para cubrir
todas las facetas del desarrollo y la implementación del software.

\subsubsection*{Pruebas unitarias}
Las pruebas unitarias se centran en verificar la funcionalidad de componentes individuales del sistema, como funciones o
métodos. Estas pruebas son cruciales para asegurar que cada parte del código funciona correctamente en aislamiento.

Se utilizan frameworks de pruebas, como PHPUnit, para automatizar estas pruebas y hacerlas repetibles y confiables.


Pruebas unitarias ejecutadas desde un entorno de desarrollo integrado

\subsubsection*{Pruebas funcionales}
Las pruebas funcionales evalúan el sistema desde el punto de vista del usuario final, verificando que las
funcionalidades del software se comportan como se espera cuando se integran varios componentes.


Pruebas funcionales ejecutadas desde la línea de comandos

\subsection*{Integración continua}
La integración continua es una práctica esencial en el desarrollo de software moderno, que implica la integración
frecuente del trabajo de los desarrolladores en un repositorio compartido, normalmente programando la ejecución
automática de los test en un servidor de integración continua, cada vez que se añade un commit al repositorio.

En este proyecto, hemos implementado la integración continua utilizando GitHub Actions, un servicio de automatización
que permite crear flujos de trabajo personalizados para compilar, probar y desplegar código directamente desde GitHub.


Pruebas de Integración continua realizada en GitHub actions.

\colorbox{color_highlight}{@TODO: @marlene:}
En cuanto a los datos, los generaste manual. No hay forma de generar datos sintéticos? Se podrían generar 100 o 1000
documentos? Sino, hay que justificarlo bien.