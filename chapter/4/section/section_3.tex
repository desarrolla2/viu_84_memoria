\section{Desarrollo de software}\label{sec:desarrollo_de_software}

Este proyecto parte de una base realizada en la asignatura 47 GIIN Proyecto de ingeniería del
software~\cite{https://github.com/desarrolla2/viu_47_proyecto_de_ingenieria_del_software} de este mismo grado
universitario.

\subsection*{Objetivos}

A partir del trabajo realizado en esa asignatura se plantean los siguientes objetivos, que ya fueron recogidos en el
anteproyecto.

\begin{enumerate}
    \item \textbf{Definir un nuevo conjunto de datos de prueba}: En el trabajo previo ya existía un conjunto de datos de
    prueba.
    Para este TFG se quería definir un nuevo conjunto, utilizando si fuera posible un conjunto de datos de fuentes
    abiertas.
    \item \textbf{Evaluar la precisión del sistema}: Para este TFG se quería evaluar como de preciso es el sistema a la
    hora de la extracción de información de los documentos.
    \item \textbf{Experimentar y seleccionar uno o más sistemas LLM}: En el trabajo previo se había probado con los
    modelos \textbf{ollama}~\cite{https://ollama.com/documentation} y
    \textbf{gpt-3.5-turbo}~\cite{https://platform.openai.com/docs/models/gpt-3.5-turbo}.
    Para este TFG se quería probar con nuevos modelos.
    Finalmente, este trabajo implicó reescribir completamente algunos componentes.
    \item \textbf{Mejora de las interfaces}: En este TFG se querían realizar mejoras en la capa de presentación de ambas
    interfaces.
    \item \textbf{Documentación formal}: elaboración de la documentación requerida para completar satisfactoriamente
    este TFG.
\end{enumerate}

Debían respetarse además los tres requisitos definidos previamente en el capítulo~\ref{ch:chapter_1} Introducción.

\subsection*{Plan de trabajo}

Una vez que se han identificado los requisitos y los objetivos se estableció un plan de trabajo, que además tal y como
se describió en la sección~\ref{sec:iterativo_incremental} Iterativo incremental, al final de cada sprint se realizó
una pequeña retrospectiva, para evaluar que tan bien estaba funcionando el plan de trabajo, así como para planificar
el siguiente sprint y modificar si fuera necesario el plan de trabajo restante en función de los resultados obtenidos
hasta ese momento.

\subsubsection{Sprint 0: Hasta el 24 de marzo}

Se realizó el planteamiento inicial del TFG, la elaboración y entrega del anteproyecto.
Se realizó la instalación del proyecto y una nueva selección de las herramientas de trabajo.

\subsubsection{Sprint 1: del 25 de marzo al 7 de abril}

Se realizó una ampliación del conjunto de datos de prueba.
Para ello se realizó una investigación sobre distintos bases de datos que ofrecen conjuntos de datos abiertos; sin
embargo, estas fuentes de datos ofrecen conjuntos de datos anonimizados, es decir datos como nombres, apellidos,
números de documentos, direcciones, no aparecen.

Es por este motivo que se descartó utilizar uno de estos conjuntos de datos.

\subsubsection{Sprint 2: del 8 de abril al 21 de abril}

En este sprint se determinó que el conjunto de datos de prueba debería realizarse manualmente.
Para ello se desarrolló un sistema que permitiría, escribir los contratos en formato \textit{Markdown} el cual es fácil
de manipular y luego sería convertido a \textit{PDF} utilizando \textbf{LaTeX} y diferentes plantillas para introducir
variabilidad en los formatos de entrada.

\colorbox{color_highlight}{@TODO: reescribir e insertar referencias}

\subsubsection{Sprint 3: del 22 de abril al 5 de mayo}

En apenas unos meses desde el final de la asignatura en la que se basa este trabajo, han aparecido una serie de nuevos
modelos.
Se hacía necesario investigar el funcionamiento de estos modelos para determinar si eran más adecuados que los modelos
que se utilizaron en dicho trabajo.

\begin{itemize}
    \item \textbf{Llama 3}~\cite{https://ollama.com/library/llama3}.
    \item \textbf{ChatGPT 4}~\cite{https://platform.openai.com/docs/models/gpt-4}
\end{itemize}

Después de realizar una serie de pruebas durante este \textit{sprint}, se determinó que si bien \textbf{Llama 3} es un
modelo mucho más avanzado con respecto a su versión anterior, \textbf{ChatGPT 4} demostró ser muy superior, por lo que
se determinó que se realizaría una nueva implementación del proyecto, utilizando este modelo.

\subsubsection{Sprint 4: del 6 de mayo al 19 de mayo}

Este \textit{sprint} estuvo enfocado en realizar una reimplementación de los procesadores
\textbf{Residential Lease Processor} y \textbf{Vehicle Sale And Purchase Processor} que introdujimos en la
sección~\ref{sec:diseno_del_sistema}.

\colorbox{color_highlight}{@TODO: reescribir e insertar referencias}

\subsubsection{Sprint 5: del 20 de mayo al 2 de junio}

Si bien durante todo el proyecto se han desarrollado las pruebas unitarias, de integración y funcionales oportunas, este
\textit{sprint} se dedicó a realizar una revisión formal de cómo de bueno era la implementación realizada en comparación
con los datos reales que contiene el conjunto de datos de prueba desarrollado.

\colorbox{color_highlight}{@TODO: reescribir e insertar referencias}

En este \textit{sprint}, se realizó una primera versión del borrador del TFG. Para ello se utilizó la herramienta web
\textbf{Google Docs}~\cite{https://www.google.com/docs/about/}.
Si bien los resultados con esta herramienta fueron muy buenos en cuanto a productividad y aspecto de los documentos
generados, no cumplía con algunos de los requisitos necesarios para la realización del TFG.

\subsubsection{Sprint 6: del 3 de junio al 16 de junio}

En este segundo sprint dedicado a la elaboración de la memoria fue necesario realizar una nueva versión de este
documento, pero esta vez utilizando \textbf{LaTeX}~\cite{https://www.latex-project.org/}, una herramienta más compleja
de utilizar pero mucho más potente a la hora de realizar este tipo de trabajos.

\subsubsection{Sprint 7: del 17 de junio al 30 de junio}

\colorbox{color_highlight}{@TODO: escribir}