\section{Implementación y programación}\label{sec:implemetacion_y_programacion}

En esta sección vamos a explicar los detalles de implementación más importantes de aquellos componentes que creemos
que es interesante que sean introducidos.

\subsection*{Componente \textit{Generator}}

Cómo introducimos en la sección~\ref{sec:diseno_del_sistema} Diseño del sistema, el elemento clave de este
componente es el \textit{PDF To Text Processor}.
Este procesador es responsable de convertir un fichero \textit{PDF} en su representación en texto plano.

Tal y como puede verse en el extracto de código de la figura~\ref{fig:chapter_4.4.pdf_to_text_processor}, el procesador
\textit{PDF To Text Processor} es simplemente un envoltorio o \textit{wrapper} que utiliza el componente
\textit{Command Runner} para realizar llamadas a \textit{Pdf To Text} en la línea de comandos.

\begin{figure}[ht]
    \begin{center}
        \includegraphics[width=\textwidth]{./chapter/4/images/chapter_4.4.pdf_to_text_processor}
        \caption{Captura del código fuente del procesador \textit{PDF To Text Processor}}
        \label{fig:chapter_4.4.pdf_to_text_processor}
    \end{center}
\end{figure}

En la figura~\ref{fig:chapter_4.4.generator_component_pdf_to_text_processor} puede verse un esquema de este
comportamiento: el motor envía el documento \textit{PDF} al \textit{PDF To Text Processor}, este a su vez lo envía a
\textit{PDF To Text} el cual genera un fichero de texto plano.
Finalmente, el \textit{PDF To Text Processor} extrae el contenido de ese fichero y lo envía de vuelta al motor, donde
comenzó todo.

\begin{figure}[ht]
    \begin{center}
        \includegraphics[width=\textwidth]{./chapter/4/images/chapter_4.4.generator_component_pdf_to_text_processor}
        \caption{Esquema de comunicación con la herramienta de la capa de infraestructura pdftotext}
        \label{fig:chapter_4.4.generator_component_pdf_to_text_processor}
    \end{center}
\end{figure}

\subsection*{Componente \textit{Reader}}

De este componente queremos destacar dos elementos: el \textit{Residential Lease Agreement Processor} y el
\textit{Vehicle Sale And Purchase Agreement Processor} responsables de extraer la información de los contratos de
alquiler de vivienda entre particulares y la compraventa de vehículos entre particulares respectivamente.

Cómo introducimos en la sección~\ref{sec:diseno_del_sistema} Diseño del sistema, el procesador
\textit{Residential Lease Agreement Processor} es responsable de transformar el texto de un contrato de
arrendamiento de vivienda entre particulares en un objeto con la información fundamental.

La implementación de este procesador es muy similar a la que hemos visto en el \textit{PDF To Text Processor}, por lo
que no vamos a incluir ni el código ni el esquema con la comunicación.

Simplemente, indicaremos que se trata de enviar peticiones sobre un documento que previamente hemos convertido en texto
plano a un sistema que lo pueda procesar, en este caso nos hemos decantado por \textit{ChatGPT 4 Turbo}~
\cite{url_openai_gpt4}

Si merece la pena profundizar en las características de las peticiones que se envían.
La parte más compleja de estos componentes es encontrar un \textit{prompt} o entrada lo suficientemente precisa para que
genere una respuesta que pueda ser utilizada a continuación.

Para encontrar una entrada suficientemente precisa, utilizamos dos técnicas:

La primera \textit{Few-shot Prompting}~\cite{article_few_shot_prompting} es una técnica en la cual se proporcionan
al modelo ejemplos de entrada y salida para que aprenda el patrón específico.

La segunda técnica es el \textit{Format-based Prompting}~\cite{article_format_based_prompting}, una técnica en la
cual se estructura el prompt para guiar al modelo a generar respuestas en un formato específico, como \textit{JSON},
listas o tablas, las cuales pueden ser luego fácilmente parseadas.
Por ejemplo para una persona se podría indicar que la respuesta debería ser similar a esto:
\textit{\{"name": "John", "surname": "Doe", "number": "12345678 A"\}}.

El primer intento de implementación se trató que el modelo respondiera con un \textit{XML}, utilizando ejemplos en
formato \textit{XSD}.
Sin embargo, \textit{ChatGPT 4} tiene limitaciones para trabajar con documentos \textit{XML}, por lo que la
implementación final se realizó utilizando modelos \textit{JSON}.

Todo esto puede verse en el ejemplo de código que aparece en la
figura~\ref{fig:chapter_4.4.residential_lease_agreement_processor}, donde se muestra el código del método que define el
\textit{prompt}, donde puede apreciarse que la construcción correcta del \textit{prompt} es importante para obtener
una respuesta adecuada.

\begin{figure}[ht]
    \begin{center}
        \includegraphics[width=\textwidth]{./chapter/4/images/chapter_4.4.residential_lease_agreement_processor}
        \caption{Captura del código fuente del procesador \textit{Residential Lease Agreement Processor}}
        \label{fig:chapter_4.4.residential_lease_agreement_processor}
    \end{center}
\end{figure}

Una vez que se obtiene la respuesta en formato \textit{JSON}, tan solo es necesario des serializar el mismo para
convertirlo en un objeto del tipo indicado, en este caso un \textit{Residential Lease Agreement} que contiene los tipos
de datos esperados para un contrato de arrendamiento de vivienda entre particulares: nombres, apellidos y
documentos de identidad de los arrendatarios y los arrendadores, la dirección del inmueble, la renta mensual y la fecha
del contrato.

Sin embargo, llegados a este punto comprobamos que la \textit{API} de \textit{ChatGPT 4} presenta dos limitaciones
que deben ser tenidas en cuenta: las peticiones son lentas y cuestan dinero.
Los tiempos de respuesta pueden variar dependiendo del tipo de documento y del estado de la api.
Los costes también pueden variar dependiendo del tipo de documento y del precio de la api en ese
momento~\cite{url_openai_api_pricing}.

A continuación una tabla obtenida con el promedio de realizar 100 peticiones de procesamiento para el documento
\textit{001/001.pdf}.

\begin{table}[h]
    \renewcommand{\arraystretch}{1.5}
    \setlength{\tabcolsep}{10pt}
    \begin{tabular}{p{0.60\textwidth} p{0.30\textwidth}}
        \toprule
        \textbf{Nombre}              & \textbf{Valor} \\
        \midrule
        Total peticiones             & 3              \\
        Tiempo total                 & 6.74 segundos  \\
        Total \textit{input tokens}  & ~6643          \\
        Total \textit{output tokens} & ~159           \\
        Precio total                 & ~0.07 euros    \\
        \bottomrule
    \end{tabular}
    \caption{Resumen de los resultados de la ejecución procesador \textit{Residential Lease Agreement Processor}}
    \label{tab:residential_lease_processor}
\end{table}

Un token en un modelo \textit{ChatGPT} es una unidad básica de texto que puede ser una palabra, parte de una palabra
o un carácter de puntuación.

Aunque los modelos \textit{ChatGPT} tienen limitaciones en cuanto al tamaño máximo del contexto, como se muestra en la
tabla~\ref{tab:chat_gpt_limits}, estas limitaciones no afectan la ejecución de este procesador~\cite{url_openai_models}.

Como ya dijimos, el post procesador \textit{Word Limit Post Processor}, descrito en la
sección~\ref{sec:diseno_del_sistema} Diseño del sistema, asume que la información relevante para este caso de uso se
encuentra dentro de las primeras mil palabras de cada documento, truncándose cualquier información que exceda este
límite.

\begin{table}[h]
    \renewcommand{\arraystretch}{1.5}
    \setlength{\tabcolsep}{10pt}
    \begin{tabular}{p{0.70\textwidth} >{\raggedleft\arraybackslash}p{0.20\textwidth}}
        \toprule
        \textbf{Modelo} & \textbf{Tokens por contexto} \\
        \midrule
        GPT-4 turbo     & 128,000 tokens               \\
        GPT-4           & 8,192 tokens                 \\
        GPT-3.5 turbo   & 16,385 tokens                \\
        \bottomrule
    \end{tabular}
    \caption{Límite de tokens por contexto para los modelos más recientes de \textit{ChatGPT}}
    \label{tab:chat_gpt_limits}
\end{table}

Cabe mencionar que durante la fase de desarrollo de este proyecto se repitieron continuamente los mismos documentos,
en un ciclo de prueba y error.
Por lo que tanto para acelerar los tiempos de ejecución como para reducir los costes se hizo recomendable implementar un
sistema de caché.

El sistema de caché genera un hash para cada petición y guarda la respuesta.
Si se repite una misma petición recupera la respuesta de la caché, por lo que la petición se resuelve de forma
prácticamente instantánea y sin ningún coste adicional.

\subsubsection*{Vehicle Sale And Purchase Agreement Processor}

Este procesador es responsable de transformar el texto de un contrato de compraventa de vehículo entre particulares en
un objeto con sus datos fundamentales.

Este procesador funciona de forma tan similar al procesador anterior que no será necesario entrar en nuevos detalles
con respecto a su funcionamiento, hasta el punto de que en el momento del desarrollo fue evidente que podrían
compartir una parte muy importante del código, simplemente cambiando ligeramente las peticiones que se debían realizar.

\subsection*{Registros}\label{subsec:logs}

El registro de \textit{logs} es una parte crucial del monitoreo y mantenimiento de cualquier aplicación.
En proyectos de este tipo, por la cantidad de pequeños componentes y por interactuar con elementos de infraestructura
externos, se hace necesario contar con un buen sistema de registro de logs.

En este proyecto, se ha implementado un sistema de registros utilizando \textit{monolog}, una biblioteca de registro
para \textit{PHP}.

Se han configurado tres canales.

\begin{itemize}
    \item \textit{Generator}: para los logs del componente \textit{generator}.
    \item \textit{Reader}: para los logs del componente \textit{reader}.
    \item \textit{HTTP}: para el componente que realiza las peticiones \textit{HTTP}.
\end{itemize}

Un canal es la forma en la que monolog, agrupa un conjunto de información para poder filtrar y procesar adecuadamente.
Además, los logs se almacenan en dos ficheros y formatos diferentes:

\begin{itemize}
    \item
    \textbf{Monolog format}, es el formato estándar de ficheros de log generados por esta biblioteca.

    \item
    \textbf{Logstash format}, es el estándar de la herramienta \textit{monolog}.
\end{itemize}

El formato \textit{Monolog format} está indicado para entornos de desarrollo o proyectos de pequeña envergadura.
Siempre que estos ficheros no sean demasiado grandes, se pueden trabajar a través de herramientas de línea de comandos
como \textit{grep} o \textit{awk}.

El formato \textit{Logstash format} es adecuado para ser procesado por un sistema compuesto por
\textit{Elasticsearch}, \textit{Logstash} y \textit{Kibana} (ELK).
Este formato está indicado en entornos de producción, donde la monitorización de los logs sea una tarea relevante.

En la figura~\ref{fig:chapter_4.4.logs_overview} puede verse un esquema de este comportamiento.

\begin{figure}[ht]
    \begin{center}
        \includegraphics[width=\textwidth]{./chapter/4/images/chapter_4.4.logs_overview}
        \caption{Esquema del sistema de procesamiento de logs}
        \label{fig:chapter_4.4.logs_overview}
    \end{center}
\end{figure}