\section{Tecnologías utilizadas}\label{sec:tecnologias_utilizadas}

A continuación, se describen las principales tecnologías utilizadas en el proyecto, categorizadas en distintas áreas
según su propósito y aplicación.

\subsection*{Servidor}

Las tecnologías de servidor o \textit{backend} son aquellas que se ejecutan en un servidor gestionado por la propia
organización.
Habitualmente se ocupan de la lógica de negocio y la persistencia de datos.

\begin{itemize}
    \item \textbf{PHP} es un lenguaje de scripting de propósito general que está especialmente diseñado para el
    desarrollo web ~\cite{url_php}.
    \item \textbf{Symfony} es un \textit{framework} PHP ampliamente utilizado para desarrollar aplicaciones web.
    Ofrece una arquitectura robusta y flexible, que facilita el desarrollo de aplicaciones mantenibles y escalables
    ~\cite{url_symfony}.
    \item \textbf{Twig} es un motor de plantillas para PHP, utilizado principalmente en proyectos basados en
    Symfony ~\cite{url_twig}.
    Un sistema de plantillas permite separar la capa de presentación de la capa de negocio.
\end{itemize}

En este proyecto hemos utilizado \textit{PHP} como lenguaje principal para desarrollar nuestro sistema.
Symfony se utiliza en la parte de infraestructura, incluida la interfaz web, para la que también hemos usado Twig.

\subsection*{Cliente}

Las tecnologías de cliente o \textit{frontend} son aquellas que se ejecutan en el navegador del usuario de una
aplicación web.
El \textit{frontend} se encarga de la presentación y la experiencia del usuario.

\begin{itemize}
    \item \textbf{HTML} es el lenguaje de marcado para crear páginas web.
    HTML introduce nuevas funcionalidades como elementos semánticos y multimedia, mejorando la accesibilidad y la
    interoperabilidad de las páginas web~\cite{url_html5}.
    \item \textbf{CSS} es la tecnología de estilos para el diseño visual de páginas web.
    CSS permite aplicar estilos complejos y responsivos a los elementos HTML, mejorando la presentación y la
    experiencia del usuario~\cite{url_css3}.
    \item \textbf{JavaScript} es el lenguaje de programación para la interactividad del lado del cliente.
    JavaScript permite crear experiencias de usuario dinámicas, manejando eventos y actualizando el contenido de la
    página sin recargar~\cite{url_javascript}.
    \item \textbf{Bootstrap} es un framework de \textit{frontend} que facilita el diseño de sitios y aplicaciones web
    responsivas y móviles.
    Bootstrap proporciona una colección de componentes CSS y JavaScript predefinidos que aceleran el desarrollo de
    interfaces de usuario ~\cite{url_bootstrap}.
    \item \textbf{Font Awesome} es una biblioteca de iconos vectoriales y herramientas que proporciona una amplia
    gama de iconos escalables y personalizables para su uso en proyectos web.
    Facilita la inclusión de iconos de alta calidad sin depender de imágenes ~\cite{url_fontawesome}.
\end{itemize}

Hemos utilizado \textit{HTML}, \textit{CSS} y \textit{JavaScript} para el desarrollo de la interfaz web.
Además, hemos usado \textit{Bootstrap} para realizar un prototipado rápido y finalmente hemos utilizado
\textit{Font Awesome} para la representación de iconos y logos.

\subsection*{Gestión de dependencias}

La gestión de dependencias es un aspecto crucial en el desarrollo de software.
Se encarga de administrar las bibliotecas y paquetes que un proyecto necesita para funcionar correctamente.

\begin{itemize}
    \item \textbf{Composer}
    es una herramienta de gestión de dependencias para PHP, que permite declarar en un fichero las bibliotecas de las
    que depende el proyecto y asegura que las versiones correctas de las bibliotecas se instalen y mantengan
    actualizadas ~\cite{url_composer}.
\end{itemize}

Hemos utilizado  \textit{Composer} para instalar todas las librerías  \textit{PHP} necesarias, principalmente
\textit{Symfony} y algunos de sus componentes.

\subsection*{Control de versiones}

El control de versiones es una práctica en el desarrollo de software que permite rastrear y gestionar los cambios
realizados en el código fuente a lo largo del tiempo.

\begin{itemize}
    \item \textbf{Git}
    es un sistema de control de versiones distribuido, diseñado para manejar desde proyectos pequeños hasta muy grandes
    con rapidez y eficiencia.
    Git permite a los desarrolladores colaborar de manera efectiva, rastreando cambios y gestionando ramas y fusiones
    ~\cite{url_git}.
    \item \textbf{GitHub} es una plataforma de hospedaje de repositorios Git.
    Proporciona herramientas para la colaboración, la revisión de código y facilitando el trabajo en equipo
    ~\cite{url_github}.
\end{itemize}

Hemos utilizado \textit{Git} yh \textit{GitHub} para gestionar las versiones del código~\cite{url_viu_84_proyecto}.
Se ha seguido el patrón de crear una nueva rama para cada nueva característica y fusionarla a la rama principal, cuando
esta está completada.
También hemos enlazado con el sistema de gestión de proyectos a través de los mensajes de cada commit.

\subsection*{Automatización}

La automatización en el desarrollo de software se refiere al uso de herramientas y scripts para ejecutar tareas
repetitivas de manera automática, sin intervención manual

\begin{itemize}
    \item \textbf{Make}
    es una herramienta de automatización de tareas que utiliza archivos Makefile para definir y ejecutar tareas de
    construcción y gestión de proyectos.
    Facilita la automatización de tareas repetitivas y la configuración del entorno de desarrollo
    ~\cite{url_make}.
\end{itemize}

Hemos usado \textit{Make} para automatizar las tareas más comunes como son la construcción o el inicio de los
contenedores de docker, pero también para tener acceso directo a tareas que resultaban repetitivas.

\subsection*{Pruebas}

Las pruebas en el desarrollo de software son un proceso fundamental para garantizar la calidad y funcionalidad de una
aplicación.
Consisten en la ejecución de diversos tipos de evaluaciones, como pruebas unitarias, de integración y de sistema, para
identificar y corregir errores.

\begin{itemize}
    \item \textbf{PHPUnit} es un marco de pruebas unitarias para PHP. Permite a los desarrolladores escribir y
    ejecutar pruebas automatizadas para asegurar que el código se comporta como se espera, facilitando el desarrollo de
    software de alta calidad~\cite{url_phpunit}.
    \item \textbf{GitHub Actions} es una plataforma de integración continua que permite automatizar flujos de
    trabajo directamente desde \textit{GitHub}.
    \textbf{GitHub Actions} facilita la configuración de pipelines de CI/CD, ejecutando pruebas y despliegues
    automáticamente con cada cambio en el repositorio ~\cite{url_github_actions}.
\end{itemize}

En la sección~\ref{sec:pruebas_y_evaluacion} Pruebas y evaluación, veremos que las pruebas son una parte fundamental
de este TFG.

Hemos utilizado \textit{PHPUnit} para la definición y ejecución de una batería de pruebas o \textit{test suite} que
contiene pruebas unitarias, pruebas de integración y pruebas funcionales.

Hemos utilizado \textit{GitHub Actions} para que esta \textit{suite} se ejecutara automáticamente después de cada
commit, en un proceso de integración continua.

\subsection*{Contenerización}

La contenerización es una técnica de virtualización del sistema operativo que permite empaquetar aplicaciones y
sus dependencias en contenedores.
Estos contenedores aseguran que las aplicaciones se ejecuten de manera consistente en cualquier entorno,
independientemente de las configuraciones del sistema subyacente.

\begin{itemize}
    \item \textbf{Docker} es una plataforma que permite desarrollar, enviar y ejecutar aplicaciones dentro de
    contenedores.
    Proporciona un entorno consistente para el desarrollo, pruebas y despliegue, asegurando que las aplicaciones
    funcionen de manera idéntica en diferentes entornos ~\cite{url_docker}.
    \item \textbf{Docker Compose} es una herramienta para definir y ejecutar aplicaciones Docker multi-contenedor.
    Permite orquestar varios servicios que componen una aplicación, facilitando la configuración y gestión de entornos
    de desarrollo complejos ~\cite{url_docker_compose}.
\end{itemize}

Hemos utilizado \textit{Docker} para definir dos conjuntos de contenedores, uno básico y otro ampliado.

El conjunto básico contiene un único contenedor en el que se ejecuta la aplicación.
El conjunto ampliado contiene servicios que son complementarios, en este caso el conjunto \textit{ELK} descrito en la
siguiente subsección.

\subsection*{Registros}\label{subsec:chapter_4.logs}

Los registros o \textit{logs} contienen información de eventos que ocurren en un sistema, son utilizados para
monitorear, diagnosticar y solucionar problemas.
Capturan información vital sobre el funcionamiento de aplicaciones, servidores y otros componentes, proporcionando un
historial que pueda ser analizado.

\begin{itemize}
    \item \textbf{Monolog} es una biblioteca de \textit{logs} para PHP. Permite enviar registros a varios destinos, como
    archivos, bases de datos y servicios de terceros, facilitando la monitorización y depuración de aplicaciones
    ~\cite{url_monolog}.
    \item \textbf{Elastic Search} es un motor de búsqueda y análisis de texto completo basado en Lucene.
    Permite almacenar, buscar y analizar grandes volúmenes de datos en tiempo real
    ~\cite{url_elasticsearch}.
    \item \textbf{Logstash} es una herramienta de procesamiento de datos que ingesta, transforma y envía datos a varios
    destinos ~\cite{url_logstash}.
    \item \textbf{Kibana} es una herramienta de visualización de datos que trabaja en conjunto con Elasticsearch.
    Permite a los usuarios crear gráficos y dashboards interactivos para visualizar y analizar los datos de logs
    almacenados en Elasticsearch ~\cite{url_kibana}.
\end{itemize}

Hemos utilizado \textit{Monolog} para configurar los \textit{logs} generados en la aplicación.
El resto de elementos, \textit{Elastic Search}, \textit{Logstash} y \textit{Kibana}, conocidos como \textit{ELK},
permiten interpretar en tiempo real los registros de la aplicación.

En la sección~\ref{sec:implemetacion_y_programacion} Implemetación y programación, ampliamos los detalles.

\subsection*{Componente \textit{Generator}}

El componente \textit{Generator} fue descrito en la subsección~\ref{subsec:chapter_4.generator_component} Componente
\textit{Generator} de la sección anterior.

\begin{itemize}
    \item \textbf{Pdf to Text} es una herramienta que convierte documentos \textit{PDF} en texto plano.
    Permite extraer el contenido textual de archivos \textit{PDF}
    , lo cual es útil para la posterior manipulación y análisis de
    datos ~\cite{url_pdftotextl}.
\end{itemize}

Utilizamos esta tecnología dentro del procesador \textit{PDF To Text Processor} para la conversión de documentos
\textit{PDF} en texto plano.

\subsection*{Componente Reader}

El componente \textit{Generator} fue descrito en la subsección~\ref{subsec:chapter_4.reader_component} Componente
\textit{Reader} de la sección anterior.

\begin{itemize}
    \item \textbf{ChatGPT API} proporciona acceso a modelos avanzados de procesamiento de lenguaje natural.
    Permite integrar capacidades de IA en la aplicación, como la generación de texto y el análisis de datos, mejorando
    las funcionalidades y experiencias del usuario ~\cite{url_openai_api_documentation}.
    \item \textbf{Symfony HttpClient} es un cliente HTTP flexible y eficiente para PHP. Permite realizar solicitudes
    HTTP a servicios externos, manejar respuestas y gestionar errores de manera sencilla y eficaz
    ~\cite{url_symfony_http}.
    \item \textbf{Symfony Cache} es un componente de Symfony que proporciona una implementación robusta y flexible
    para el almacenamiento en caché ~\cite{url_symfony_cache}.
    Permite mejorar el rendimiento de la aplicación mediante el almacenamiento temporal de datos, reduciendo la carga en
    los recursos externos.
\end{itemize}

Hemos utilizado la \textit{API} de \textit{ChatGPT}, para extraer la información de los documentos en los procesadores
\textit{Residential Lease Agreement Processor} y \textit{Vehicle Sale And Purchase Agreement Processor}.

Las llamadas \textit{HTTP} se realizaron mediante la librería \textit{Symfony HttpClient} y finalmente
\textit{Symfony Cache} cachea las respuestas.
Veremos más en profundidad este proceso en la sección~\ref{sec:implemetacion_y_programacion} Implemetación y
programación.



