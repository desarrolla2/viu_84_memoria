\chapter{Desarrollo}\label{ch:chapter_3}


\section{Descripción del proyecto}

Hemos creado una solución tecnológica con dos interfaces una web y una de línea de comandos, que transforma los datos no
estructurados de documentos PDF en formatos estructurados y utilizables.


\section{Alcance}

El software que hemos desarrollado hasta ahora sirve como una base conceptual y prototipo inicial.

Se ha desarrollado una herramienta que es capaz de procesar datos de contratos de los siguientes tipos

\begin{enumerate}
    \item Contrato de arrendamiento de vivienda
    \item Contrato de compraventa de vehículo entre particulares
\end{enumerate}

Aunque ha demostrado la viabilidad de la idea, aún requiere validación extensiva y mejoras antes de que esté listo para
su implementación en entornos de producción real.

Nuestro próximo objetivo es abrir el proyecto a la comunidad, convirtiéndolo en una iniciativa de código abierto.

Esto no solo incluirá la mejora y ampliación de la documentación existente, sino también la traducción al inglés para
facilitar su adopción y contribución global.
Además, planeamos desarrollar más funcionalidades y realizar pruebas exhaustivas para asegurar la robustez y fiabilidad
del software.


\section{Metodología de desarrollo}

Nuestro proyecto se ha desarrollado utilizando la metodología Extreme
Programming (XP), una metodología ágil que enfatiza la adaptabilidad y la colaboración del equipo (Beck, 2000).

Hemos adoptado prácticas como la integración continua y el desarrollo orientado a pruebas para mejorar la
calidad del código y la respuesta a los cambios (Fowler \& Foemmel, 2006).

Hemos gestionado el flujo de trabajo utilizando un tablero Kanban en GitHub Projects, lo que ha permitido un seguimiento
visual y eficiente del progreso (Anderson, 2010).


\section{Tecnologías utilizadas}
El proyecto hace uso de una variedad de tecnologías modernas para asegurar un desarrollo eficiente,
una implementación robusta y una experiencia de usuario óptima.

A continuación, se describen las principales tecnologías utilizadas en
el proyecto, categorizadas en distintas áreas según su propósito y aplicación.

\subsection{Backend}

*PHP

Es un lenguaje de scripting de propósito general que está especialmente diseñado para el desarrollo web.
PHP permite crear aplicaciones web dinámicas y está ampliamente soportado en servidores web.

Symfony

Symfony es un framework PHP ampliamente utilizado para desarrollar aplicaciones web.
Ofrece una arquitectura robusta y flexible, que facilita el desarrollo de aplicaciones mantenibles y escalables.

Twig

Twig es un motor de plantillas para PHP, utilizado principalmente en proyectos basados en
Symfony.
Facilita la creación de vistas dinámicas y mantenibles, separando la lógica de presentación del código de negocio.

\subsection{Frontend}

HTML5

Es el estándar de marcado para crear páginas web.
HTML5 introduce nuevas funcionalidades como elementos semánticos y multimedia, mejorando la accesibilidad y la
interoperabilidad de las páginas web.

CSS3

Es la tecnología de estilos para el diseño visual.
CSS3 permite aplicar estilos complejos y responsivos a los elementos HTML, mejorando la presentación y la experiencia
del usuario.

JavaScript

Es el lenguaje de programación para la interactividad del lado del cliente.
JavaScript permite crear experiencias de usuario dinámicas y responsivas, manejando eventos y actualizando el contenido
de la página sin recargar.

Bootstrap

Es un framework de front-end que facilita el diseño de sitios y aplicaciones web
responsivas y móviles.
Bootstrap proporciona una colección de componentes CSS
y JavaScript predefinidos que aceleran el desarrollo de interfaces de usuario atractivas.

Font Awesome

Es una biblioteca de iconos vectoriales y herramientas que proporciona una amplia gama de
iconos escalables y personalizables para su uso en proyectos web.
Facilita la inclusión de iconos de alta calidad sin depender de imágenes.

\subsection{Gestión de dependencias}

Composer

Es una herramienta de gestión de dependencias para PHP, que
permite declarar las bibliotecas de las que depende tu proyecto
y las gestiona.
Composer asegura que las versiones correctas de las bibliotecas se instalen y mantengan actualizadas.

\subsection{Control de versiones}

Git

Es un sistema de control de versiones distribuido, diseñado para manejar todo, desde proyectos
pequeños hasta muy grandes con rapidez y eficiencia.
Git permite a los desarrolladores colaborar de manera efectiva, rastreando cambios y gestionando ramas y fusiones.


GitHub

GitHub es una plataforma de hospedaje de repositorios Git.
Proporciona herramientas para la colaboración, la revisión de código y la gestión de proyectos, facilitando el trabajo
en equipo y la integración continua.

\subsection{Automatización}

Make

Make es una herramienta de automatización de tareas que utiliza archivos Makefile para definir y ejecutar tareas de
construcción y gestión de proyectos.
Facilita la automatización de tareas repetitivas y la configuración del entorno de desarrollo.

\subsection{Pruebas}

PHPUnit

PHPUnit es un marco de pruebas unitarias para PHP. Permite a los desarrolladores escribir y ejecutar pruebas
automatizadas
para asegurar que el código se comporta como se espera, facilitando el desarrollo de software de alta calidad.

GitHub Actions

Es una plataforma de integración continua que permite automatizar flujos de trabajo directamente desde GitHub.
GitHub Actions facilita la configuración de pipelines de CI/CD, ejecutando pruebas y despliegues automáticamente con
cada cambio en el repositorio.

\subsection{Contenerización}

Docker

Docker es una plataforma que permite desarrollar, enviar y ejecutar aplicaciones dentro de
contenedores.
Proporciona un entorno consistente para el desarrollo, pruebas y despliegue, asegurando que las aplicaciones funcionen
de manera idéntica en diferentes entornos.

Docker Compose

Docker Compose es una herramienta para definir y ejecutar aplicaciones Docker multi-contenedor.
Permite orquestar varios servicios que componen una aplicación, facilitando
la configuración y gestión de entornos de desarrollo complejos.

\subsection{Logs}

Monolog

Monolog es una biblioteca de logging para PHP. Permite enviar registros a varios destinos, como archivos, bases de datos
y servicios de terceros, facilitando la monitorización y depuración de aplicaciones.

Elastic Search

Elasticsearch es un motor de búsqueda y análisis de texto completo basado en Lucene.
Permite almacenar, buscar y analizar grandes volúmenes de datos en tiempo real.

Logstash

Logstash es una herramienta de procesamiento de datos que ingesta, transforma y envía datos a varios
destinos.
Es parte del paquete Elasticsearch, Logstash, Kibana (ELK) y facilita la recolección y procesamiento de logs.

Kibana

Kibana es una herramienta de visualización de datos que trabaja en conjunto con Elasticsearch.
Permite a los usuarios crear gráficos y dashboards interactivos para visualizar y analizar los datos de logs almacenados
en Elasticsearch.

\subsection{Componente Generator}

Pdf to text

Pdf to Text es una herramienta que convierte documentos PDF en texto plano.
Permite extraer el contenido textual de archivos PDF, lo cual es útil para la posterior manipulación y análisis de
datos.

\subsection{Componente Reader}

Symfony HttpClient

Symfony HttpClient es un cliente HTTP flexible y eficiente para PHP. Permite realizar solicitudes HTTP a servicios
externos, manejar respuestas y gestionar errores de manera sencilla y eficaz.

Symfony Cache

Symfony Cache es un componente de Symfony que proporciona una implementación robusta y flexible para el almacenamiento
en caché.
Permite mejorar el rendimiento de la aplicación mediante el almacenamiento temporal de datos, reduciendo la carga en los
recursos externos.

API Open AI

La API de OpenAI proporciona acceso a modelos avanzados de procesamiento de lenguaje natural.
Permite integrar capacidades de IA en la aplicación, como la generación de texto y el análisis de datos, mejorando las
funcionalidades y experiencias del usuario.
