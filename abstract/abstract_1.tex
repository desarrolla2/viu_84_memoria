\newpage
\section*{Resumen}
\addcontentsline{toc}{section}{Resumen}
El presente trabajo se centra en el desarrollo de un sistema de extracción de información de documentos utilizando
tecnologías avanzadas y enfoques modernos de arquitectura de software.
El objetivo principal del proyecto es crear una herramienta eficiente y precisa para convertir documentos PDF en texto y
extraer datos relevantes, aplicando los principios de ``código limpio'' y ``arquitectura limpia'' para asegurar la
mantenibilidad y calidad del código.

Para el desarrollo del sistema, se utilizó un enfoque modular, implementando componentes: el \textit{Generator}, que
convierte documentos PDF en texto, y el \textit{Reader}, que extrae la información requerida del texto generado.

En términos de rendimiento, se constató que el núcleo del sistema es rápido, aunque que el rendimiento general depende
de las herramientas seleccionadas para la capa de infraestructura.
Las pruebas de rendimiento mostraron tiempos de ejecución bajos para herramientas locales como \textbf{PDFToText}, pero
tiempos más elevados para la API de \textbf{ChatGPT}.

El análisis de los resultados demostró una precisión del 100\% en la identificación y extracción de datos para los
conjuntos de prueba utilizados.
Sin embargo, dado que \textbf{ChatGPT} es un sistema no determinista y la muestra es relativamente pequeña, es posible
que los resultados reales sean ligeramente inferiores.

Las implicaciones de este trabajo son significativas tanto en el ámbito académico como profesional.
En el ámbito académico, el proyecto muestra un nuevo enfoque para aprovechar los sistemas LLM.
Profesionalmente, demuestra el potencial de estas tecnologías para transformar la forma en que las empresas y
organizaciones gestionan la documentación.

\vspace{1cm}

\textbf{Palabras clave}: Extracción de información, código limpio, arquitectura limpia, Symfony, PHP, LLM, ChatGPT.