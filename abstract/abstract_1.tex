\newpage
\section*{Resumen}
\addcontentsline{toc}{section}{Resumen}

El presente trabajo se centra en el desarrollo de un sistema de extracción de información de documentos de cualquier
tipo y formato, utilizando tecnologías avanzadas y enfoques modernos de arquitectura de software.

Dado que este objetivo general puede resultar demasiado ambicioso para un TFG, el objetivo específico será crear una
herramienta con capacidad para procesar documentos PDF, enfocándose en dos tipos de documentos concretos: los contratos
de arrendamiento de vivienda entre particulares y los contratos de compraventa de vehículos entre particulares.

Aplicando los principios de código limpio y arquitectura limpia, se podrá asegurar la calidad del
código y la capacidad para añadir soporte a nuevos formatos de documento y a nuevos tipos de contratos.

Para el desarrollo del sistema, se utilizó un enfoque modular, implementando dos componentes principales: el
\textit{Generator}, que convierte documentos PDF en texto, y el \textit{Reader}, que extrae la información requerida
del texto generado.

El análisis de los resultados demostró una precisión del 100\% en la identificación y extracción de datos para los
conjuntos de prueba utilizados.
Sin embargo, dado que los modelos LLM como \textit{ChatGPT} no son sistemas deterministas y la muestra es relativamente
pequeña, es posible que los resultados reales sean ligeramente inferiores.

En términos de rendimiento, se constató que el núcleo del sistema es rápido, aunque el rendimiento general depende de
las herramientas seleccionadas para la capa de infraestructura.
Las pruebas de rendimiento mostraron tiempos de ejecución bajos para herramientas locales como \textit{PDF To Text}
, pero
tiempos más elevados para la \textit{API} de \textit{ChatGPT}.


\vspace{1cm}

\textbf{Palabras clave}: extracción de información, código limpio, arquitectura limpia, PHP, Symfony, PDF, LLM, ChatGPT.