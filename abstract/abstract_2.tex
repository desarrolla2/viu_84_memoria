\newpage
\section*{Abstract}
\addcontentsline{toc}{section}{Abstract}
The present work focuses on the development of a document information extraction system using advanced technologies and
modern software architecture approaches.
The main objective of the project is to create an efficient and precise tool for converting PDF documents into text and
extracting relevant data, applying the principles of ``clean code'' and ``clean architecture'' to ensure code
maintainability and quality.

For the development of the system, a modular approach was used, implementing components: the \textit{Generator}, which
converts PDF documents into text, and the \textit{Reader}, which extracts the required information from the generated
text.

In terms of performance, it was found that the core of the system is fast, although the overall performance depends on
the tools selected for the infrastructure layer.
Performance tests showed low execution times for local tools such as \textbf{PDFToText}, but higher times for the
\textbf{ChatGPT} API.

The analysis of the results demonstrated 100\% accuracy in identifying and extracting data for the test sets used.
However, given that \textbf{ChatGPT} is a non-deterministic system and the sample is relatively small, it is
possible that the actual results may be slightly lower.

The implications of this work are significant both academically and professionally.
Academically, the project shows a new approach to leveraging LLM systems.
Professionally, it demonstrates the potential of these technologies to transform the way companies and organizations
manage documentation.

\vspace{1cm}

\textbf{Keywords}: Information extraction, clean code, clean architecture, Symfony, PHP, LLM, ChatGPT.

